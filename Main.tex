\documentclass[a4paper,13pt,twoside]{extreport}
\usepackage[utf8]{vietnam}
\usepackage[top=2cm, bottom=2cm, left=3.5cm, right=2.5cm]{geometry}

\usepackage{graphicx} % Cho phép chèn hỉnh ảnh
\usepackage{fancybox} % Tạo khung box
\usepackage{indentfirst} % Thụt đầu dòng ở dòng đầu tiên trong đoạn
\usepackage{amsthm} % Cho phép thêm các môi trường định nghĩa
\usepackage{latexsym} % Các kí hiệu toán học
\usepackage{amsmath} % Hỗ trợ một số biểu thức toán học
\DeclareMathOperator*{\argmax}{arg\,max}
\usepackage{amssymb} % Bổ sung thêm kí hiệu về toán học
\usepackage{amsbsy} % Hỗ trợ các kí hiệu in đậm
\usepackage{times} % Chọn font Time New Romans
\usepackage{array} % Tạo bảng array
\usepackage{enumitem} % Cho phép thay đổi kí hiệu của list
\usepackage{subfiles} % Chèn các file nhỏ, giúp chia các chapter ra nhiều file hơn
\usepackage{titlesec} % Giúp chỉnh sửa các tiêu đề, đề mục như chương, phần,..
\usepackage{titletoc}
\usepackage{chngcntr} % Dùng để thiết lập lại cách đánh số caption,..
\usepackage{pdflscape} % Đưa các bảng có kích thước đặt theo chiều ngang giấy
\usepackage{afterpage}
\usepackage[ruled,vlined]{algorithm2e}  % Hỗ trợ viết các giải thuật
\usepackage{capt-of} % Cho phép sử dụng caption lớn đối với landscape page
\usepackage{multirow} % Merge cells
\usepackage{fancyhdr} % Cho phép tùy biến header và footer
% \usepackage[natbib,backend=biber,style=ieee]{biblatex} % Giúp chèn tài liệu tham khảo
\usepackage{appendix}
\usepackage{outlines}
\usepackage[font=small,labelfont=bf]{caption}

\usepackage{listings}
\usepackage{inconsolata}  % For code font
\usepackage{pythonhighlight}

\usepackage{float}
\usepackage{subcaption}
\usepackage{xurl}

\usepackage[nonumberlist, nopostdot, nogroupskip, acronym]{glossaries}
\usepackage{glossary-superragged}
\setglossarystyle{superraggedheaderborder}
\usepackage{setspace}
\usepackage{parskip}

% package content table
\usepackage{tocbasic}

\usepackage{blindtext}

% Add tcolorbox package for shaded boxes
\usepackage[most]{tcolorbox}

% Add packages for flowchart
\usepackage{tikz}
\usetikzlibrary{shapes,arrows,positioning,calc,arrows.meta}
\usepackage{pgfplots}
\pgfplotsset{compat=1.18}
\usepackage{csquotes}

% Define a reusable shaded box style
\newtcolorbox{condensed_idea}[1][]{
    enhanced,
    colback=gray!5,
    colframe=gray!50,
    arc=0mm,
    boxrule=0.5pt,
    leftrule=3pt,
    left=10pt,
    right=10pt,
    top=10pt,
    bottom=10pt,
    title=#1,
    coltitle=black,
    fonttitle=\bfseries
}

% Define HTML language for code highlighting
% CSS
\usepackage{color}
\definecolor{lightgray}{rgb}{0.95, 0.95, 0.95}
\definecolor{darkgray}{rgb}{0.4, 0.4, 0.4}
%\definecolor{purple}{rgb}{0.65, 0.12, 0.82}
\definecolor{editorGray}{rgb}{0.95, 0.95, 0.95}
\definecolor{editorOcher}{rgb}{1, 0.5, 0} % #FF7F00 -> rgb(239, 169, 0)
\definecolor{editorGreen}{rgb}{0, 0.5, 0} % #007C00 -> rgb(0, 124, 0)
\definecolor{orange}{rgb}{1,0.45,0.13}		
\definecolor{olive}{rgb}{0.17,0.59,0.20}
\definecolor{brown}{rgb}{0.69,0.31,0.31}
\definecolor{purple}{rgb}{0.38,0.18,0.81}
\definecolor{lightblue}{rgb}{0.1,0.57,0.7}
\definecolor{lightred}{rgb}{1,0.4,0.5}
\usepackage{upquote}

% ===================================================

\renewcommand{\bibname}{reference} 
\usepackage[backend=bibtex,style=ieee]{biblatex}  %backend=biber is 'better'

\renewcommand\appendixname{APPENDIX}
\renewcommand\appendixpagename{APPENDIX}
\renewcommand\appendixtocname{APPENDIX}

\renewcommand{\figurename}{Figure}
\renewcommand{\tablename}{Table}
\renewcommand{\chaptername}{CHAPTER}

\addbibresource{reference.bib} % chèn file chứa danh mục tài liệu tham khảo vào 

% Configuration for code listings
\lstset{
    basicstyle=\ttfamily\small,
    breaklines=true,
    breakatwhitespace=true,
    showstringspaces=false,
    frame=single,
    numbers=left,
    numberstyle=\tiny,
    stepnumber=1,
    numbersep=5pt,
    backgroundcolor=\color{lightgray},
    commentstyle=\color{editorGreen},
    keywordstyle=\color{blue},
    stringstyle=\color{orange},
    identifierstyle=\color{black}
}

% Define language-specific styles
\lstdefinestyle{Python}{
    language=Python,
    basicstyle=\ttfamily\small,
    keywordstyle=\color{blue},
    commentstyle=\color{editorGreen},
    stringstyle=\color{orange},
    numbers=left,
    numberstyle=\tiny,
    stepnumber=1,
    numbersep=5pt,
    backgroundcolor=\color{lightgray},
    frame=single,
    breaklines=true,
    breakatwhitespace=true,
    showstringspaces=false
}

\lstdefinestyle{JavaScript}{
    language=JavaScript,
    basicstyle=\ttfamily\small,
    keywordstyle=\color{blue},
    commentstyle=\color{editorGreen},
    stringstyle=\color{orange},
    numbers=left,
    numberstyle=\tiny,
    stepnumber=1,
    numbersep=5pt,
    backgroundcolor=\color{lightgray},
    frame=single,
    breaklines=true,
    breakatwhitespace=true,
    showstringspaces=false
}

\lstdefinestyle{HTML}{
    language=HTML,
    basicstyle=\ttfamily\small,
    keywordstyle=\color{blue},
    commentstyle=\color{editorGreen},
    stringstyle=\color{orange},
    numbers=left,
    numberstyle=\tiny,
    stepnumber=1,
    numbersep=5pt,
    backgroundcolor=\color{lightgray},
    frame=single,
    breaklines=true,
    breakatwhitespace=true,
    showstringspaces=false
} 
  % Phần này cho phép chèn code và formatting code như C, C++, Python

%\makeglossaries
\makenoidxglossaries

% New abbreviations from the table
\newglossaryentry{AI}{
    type=\acronymtype,
    name={AI},
    description={Artificial Intelligence},
    first={Artificial Intelligence (AI)}
}
\newglossaryentry{RAG}{
    type=\acronymtype,
    name={RAG},
    description={Retrieval Augmented Generation},
    first={Retrieval Augmented Generation (RAG)}
}
\newglossaryentry{AWC}{
    type=\acronymtype,
    name={AWC},
    description={Agentic Workflow Core},
    first={Agentic Workflow Core (AWC)}
}
\newglossaryentry{TE}{
    type=\acronymtype,
    name={TE},
    description={Tool Ecosystem},
    first={Tool Ecosystem (TE)}
}
\newglossaryentry{HITL}{
    type=\acronymtype,
    name={HITL},
    description={Human-in-the-loop},
    first={Human-in-the-loop (HITL)}
}
\newglossaryentry{ELMS}{
    type=\acronymtype,
    name={ELMS},
    description={Elearning Management System},
    first={Elearning Management System (ELMS)}
}
\newglossaryentry{MCP}{
    type=\acronymtype,
    name={MCP},
    description={Model Context Protocol},
    first={Model Context Protocol (MCP)}
}
\newglossaryentry{OCR}{
    type=\acronymtype,
    name={OCR},
    description={Optical Character Recognition},
    first={Optical Character Recognition (OCR)}
}
\newglossaryentry{LLM}{
    type=\acronymtype,
    name={LLM},
    description={Large Language Model},
    first={Large Language Model (LLM)}
}
\newglossaryentry{OpenAPI2MCP}{
    type=\acronymtype,
    name={OpenAPI2MCP},
    description={OpenAPI to Model Context Protocol},
    first={OpenAPI to Model Context Protocol (OpenAPI2MCP)}
}
\newglossaryentry{FFN}{
    type=\acronymtype,
    name={FFN},
    description={Feedforward Network},
    first={Feedforward Network (FFN)}
}
\newglossaryentry{MoE}{
    type=\acronymtype,
    name={MoE},
    description={Mixture of Experts},
    first={Mixture of Experts (MoE)}
}
\newglossaryentry{MaAS}{
    type=\acronymtype,
    name={MaAS},
    description={Multi-Agent Architecture Search},
    first={Multi-Agent Architecture Search (MaAS)}
}
\newglossaryentry{KC}{
    type=\acronymtype,
    name={KC},
    description={Knowledge Citation},
    first={Knowledge Citation (KC)}
}
\newglossaryentry{IDE}{
    type=\acronymtype,
    name={IDE},
    description={Integrated Development Environment},
    first={Integrated Development Environment (IDE)}
}
\newglossaryentry{RBAC}{
    type=\acronymtype,
    name={RBAC},
    description={Role-Based Access Control},
    first={Role-Based Access Control (RBAC)}
}
\newglossaryentry{BDG}{
    type=\acronymtype,
    name={BDG},
    description={Benchmarking Dataset Generation},
    first={Benchmarking Dataset Generation (BDG)}
}
\newglossaryentry{NLTK}{
    type=\acronymtype,
    name={NLTK},
    description={Natural Language Toolkit},
    first={Natural Language Toolkit (NLTK)}
}
\newglossaryentry{CoT}{
    type=\acronymtype,
    name={CoT},
    description={Chain of Thought},
    first={Chain of Thought (CoT)}
}
\newglossaryentry{CORS}{
    type=\acronymtype,
    name={CORS},
    description={Cross-Origin Resource Sharing},
    first={Cross-Origin Resource Sharing (CORS)}
}
\newglossaryentry{CSRF}{
    type=\acronymtype,
    name={CSRF},
    description={Cross-Site Request Forgery},
    first={Cross-Site Request Forgery (CSRF)}
}
\newglossaryentry{JWT}{
    type=\acronymtype,
    name={JWT},
    description={JSON Web Token},
    first={JSON Web Token (JWT)}
}
\newglossaryentry{LMS}{
    type=\acronymtype,
    name={LMS},
    description={Learning Management System},
    first={Learning Management System (LMS)}
}
\newglossaryentry{CMS}{
    type=\acronymtype,
    name={CMS},
    description={Course Management System},
    first={Course Management System (CMS)}
}
\newglossaryentry{DAG}{
    type=\acronymtype,
    name={DAG},
    description={Directed Acyclic Graph},
    first={Directed Acyclic Graph (DAG)}
}

% ===================================================


\fancypagestyle{plain}{%
\fancyhf{} % clear all header and footer fields
\fancyfoot[RO,RE]{\thepage} %RO=right odd, RE=right even
\renewcommand{\headrulewidth}{0pt}
\renewcommand{\footrulewidth}{0pt}}

\setlength{\headheight}{10pt}

\def \TITLE{GRADUATION THESIS}
\def \AUTHOR{NGUYEN DOAN ABC}

% ===================================================
\titleformat{\chapter}[hang]{\centering\bfseries}{CHƯƠNG \thechapter.\ }{0pt}{}[]

\titleformat 
    {\chapter} % command
    [hang] % shape
    {\centering\bfseries} % format
    {CHAPTER \thechapter.\ } % label
    {0pt} %sep
    {} % before
    [] % after
\titlespacing*{\chapter}{0pt}{-20pt}{20pt}

\titleformat
    {\section} % command
    [hang] % shape
    {\bfseries} % format
    {\thechapter.\arabic{section}\ \ \ \ } % label
    {0pt} %sep
    {} % before
    [] % after
\titlespacing{\section}{0pt}{\parskip}{0.5\parskip}

\titleformat
    {\subsection} % command
    [hang] % shape
    {\bfseries} % format
    {\thechapter.\arabic{section}.\arabic{subsection}\ \ \ \ } % label
    {0pt} %sep
    {} % before
    [] % after
\titlespacing{\subsection}{0pt}{\parskip}{0.5\parskip}

\titleformat
    {\subsubsection} % command
    [hang] % shape
    {\bfseries} % format
    {\thechapter.\arabic{section}.\arabic{subsection}.\arabic{subsubsection}\ \ \ \ } % label
    {0pt} %sep
    {} % before
    [] % after
\titlespacing{\subsubsection}{0pt}{\parskip}{0.5\parskip}

% \newcommand{\titlesize}{\fontsize{18pt}{23pt}\selectfont}
% \newcommand{\subtitlesize}{\fontsize{16pt}{21pt}\selectfont}
% \titleclass{\part}{top}
% \titleformat{\part}[display]
%   {\normalfont\huge\bfseries}{\centering}{20pt}{\Huge\centering}
% \titlespacing{\part}{0pt}{em}{1em}
% \titlespacing{\section}{0pt}{\parskip}{0.5\parskip}
% \titlespacing{\subsection}{0pt}{\parskip}{0.5\parskip}
% \titlespacing{\subsubsection}{0pt}{\parskip}{0.5\parskip}



% ===================================================
\usepackage{hyperref}
\hypersetup{pdfborder = {0 0 0}} %
\hypersetup{pdftitle={\TITLE},
	pdfauthor={\AUTHOR}}
	
\usepackage[all]{hypcap} % Cho phép tham chiếu chính xác đến hình ảnh và bảng biểu

\graphicspath{{figures/}{../figures/}} % Thư mục chứa các hình ảnh

\counterwithin{figure}{chapter} % Đánh số hình ảnh kèm theo chapter. Ví dụ: Hình 1.1, 1.2,..

\title{\bf \TITLE}
\author{\AUTHOR}

\setcounter{secnumdepth}{3} % Cho phép subsubsection trong report
% \setcounter{tocdepth}{3} % Chèn subsubsection vào bảng mục lục

\theoremstyle{definition}
\newtheorem{example}{Ví dụ}[chapter] % Định nghĩa môi trường ví dụ

\onehalfspacing
%Khoảng cách xuống dòng
\setlength{\parskip}{6pt}
%Lùi đầu dòng
\setlength{\parindent}{15pt}



% =========================== BODY ===============
\begin{document}
% \newgeometry{top=2cm, bottom=2cm, left=2cm, right=2cm}
\subfile{cover} % Phần bìa
% \restoregeometry

% ===================================================
\pagenumbering{roman}
\renewcommand{\figurename}{Figure}
\renewcommand{\tablename}{Table}
\renewcommand{\chaptername}{CHAPTER}
% \pagestyle{empty} % Header và footer rỗng
%\newpage
%\subfile{chapters/0_1_subject.tex}

\newpage
\subfile{Chapter/0_2_acknowledgment.tex}

\newpage
\subfile{Chapter/0_3_abstract.tex}

% ===================================================
% \pagestyle{empty} % Header và footer rỗng
\newpage
\pagenumbering{roman} % Xóa page numbering ở cuối trang
\renewcommand*\contentsname{TABLE OF CONTENT}

\titlecontents{chapter}
[0.0cm]             % left margin
{\bfseries\vspace{0.3cm}}                  % above code
{{\bfseries{\scshape}
			CHAPTER \thecontentslabel.\ }}
% numbered format
{}         % unnumbered format
{\titlerule*[0.3pc]{.}\contentspage}         % filler-page-format, e.g dots


\titlecontents{section}
[0.0cm]             % left margin
{\vspace{0.3cm}}                  % above code
{\thecontentslabel \ } % numbered format
{}         % unnumbered format
{\titlerule*[0.3pc]{.}\contentspage}         % filler-page-format, e.g dots

\titlecontents{subsection}
[0.5cm]             % left margin
{\vspace{0.3cm}}                  % above code
{\thecontentslabel \ } % numbered format
{}         % unnumbered format
{\titlerule*[0.3pc]{.}\contentspage}         % filler-page-format, e.g dots

% \titlecontents{subsubsection}
% [1.0cm]             % left margin
% {\vspace{0.3cm}}                  % above code
% {\thecontentslabel \ } % numbered format
% {}         % unnumbered format
% {\titlerule*[0.3pc]{.}\contentspage}         % filler-page-format, e.g dots

% Tạo mục lục tự động
\addtocontents{toc}{\protect\thispagestyle{empty}}
\tableofcontents
\thispagestyle{empty}
\cleardoublepage

% \pagenumbering{roman}
%Tạo danh mục hình vẽ.
\renewcommand{\listfigurename}{LIST OF FIGURES}
{\let\oldnumberline\numberline
	\renewcommand{\numberline}{Figure~\oldnumberline}
	\listoffigures}
% \phantomsection\addcontentsline{toc}{section}{\numberline {} DANH MỤC HÌNH VẼ}
\newpage


%Tạo danh mục bảng biểu.
\renewcommand{\listtablename}{LIST OF TABLES}
{\let\oldnumberline\numberline
	\renewcommand{\numberline}{Table~\oldnumberline}
	\listoftables}
% \phantomsection\addcontentsline{toc}{section}{\numberline {} DANH MỤC BẢNG BIỂU}

\glsaddall
% \renewcommand*{\glossaryname}{Danh sách thuật ngữ}
\renewcommand*{\acronymname}{LIST OF ABBREVIATIONS}
\renewcommand*{\entryname}{Abriviation}
\renewcommand*{\descriptionname}{Full Expression}
% \printnoidxglossaries
% \phantomsection\addcontentsline{toc}{section}{\numberline {} DANH MỤC THUẬT NGỮ VÀ TỪ VIẾT TẮT}
\subfile{Chapter/0_5_Danh_muc_viet_tat.tex}

% \newpage
% \subfile{Chapter/0_6_Thuat_ngu.tex}
% ===================================================


\newpage
\pagenumbering{arabic}

\pagestyle{fancy}
\fancyhf{}
\fancyhead[RE, LO]{\leftmark}
%\fancyhead[LE]{\rightmark}
\fancyfoot[RE, LO]{\thepage}

\chapter{PREFACE}
\label{chapter:Preface}
\subfile{Chapter/1_Preface}

\newpage
\pagestyle{fancy}
\chapter{LITERATURE REVIEW}
\label{chapter:Literature_review}
\subfile{Chapter/2_Literature_review}

\newpage
\pagestyle{fancy}
\chapter{DATA COLLECTION, RESEARCH AND SURVEY}
\label{chapter:Data_collection_research_and_survey}
\subfile{Chapter/3_Data_collection_research_and_survey}

\newpage
\pagestyle{fancy}
\chapter{RESEARCH MODEL \& HYPOTHESIS DEVELOPMENT}
\label{chapter:Research_model_and_hypothesis_development}
\subfile{Chapter/4_Research_model_and_hypothesis_development}

% \newpage
% \pagestyle{fancy}
% \chapter{ACHIEVED RESULTS}
% \label{chapter:Achieved_results}
% \subfile{Chapter/6_Achieved_results}

\newpage
\pagestyle{fancy}
\chapter{CONTRIBUTION, CONCLUSION AND FUTURE WORK}
\label{chapter:Contribution_conclusion_and_future_work}
% \subfile{Chapter/7_Contribution_conclusion_and_future_work}


% ===================================================
\newpage
\renewcommand\bibname{REFERENCE}
\label{chapter:Reference}
\printbibliography
\phantomsection\addcontentsline{toc}{chapter}{REFERENCE}

\appendixpage
\appendices
\addappheadtotoc
\renewcommand{\figurename}{Figure}
\renewcommand{\tablename}{Table}
\renewcommand{\chaptername}{CHAPTER}

% \chapter*{PHỤ LỤC}

%\mainmatter
\titleformat{\chapter}[hang]{\centering\bfseries}{ \thechapter.\ }{0pt}{}[]
\titlespacing*{\chapter}{0pt}{-20pt}{20pt}

\titlecontents{chapter}
[0.0cm]             % left margin
{\bfseries\vspace{0.3cm}}                  % above code
{{\bfseries{\scshape} \thecontentslabel.\ }} % numbered format
{}         % unnumbered format
{\titlerule*[0.3pc]{.}\contentspage}         % filler-page-format, e.g dots
% \chapter{DETAILED IMPLEMENTATION OF AGENTIC WORKFLOW FRAMEWORK}
% \label{appendix:a_agentic_workflow_framework}
% \subfile{Chapter/Appendix/Appendix_A}
% \newpage

% \chapter{DETAILED IMPLEMENTATION OF OPENEDX BACKEND}
% \label{appendix:b_openedx_backend}
% \subfile{Chapter/Appendix/Appendix_B}
% \newpage

% \chapter{DETAILED IMPLEMENTATION OF ELEARNING FRONTEND}
% \label{appendix:c_elearning_frontend}
% \subfile{Chapter/Appendix/Appendix_C}
% \newpage

% \chapter{DETAILED IMPLEMENTATION OF DEVELOPER TOOL}
% \label{appendix:d_developer_tool}
% \subfile{Chapter/Appendix/Appendix_D}
% \newpage

% \chapter{DETAILED IMPLEMENTATION OF UI COMPONENTS}
% \label{appendix:e_ui_components}
% \subfile{Chapter/Appendix/Appendix_E}
% \newpage
\end{document}
