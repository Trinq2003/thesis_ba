\documentclass[../Main.tex]{subfiles}
\begin{document}
	\begin{center}
		\Large{\textbf{ABSTRACT}}\\
	\end{center}
	\vspace{1cm}
	The rapid growth of Vietnam's entrepreneurial ecosystem faces significant challenges due to limited venture capital availability and complex regulatory frameworks. These constraints particularly impact early-stage startups, which often struggle to access essential resources and support systems necessary for sustainable growth. The effectiveness of University-Based Incubation Programs (UBIs) in addressing these challenges remains underexplored, despite their potential to serve as crucial catalysts for startup development in emerging markets.

	This research proposes a comprehensive investigation into the role and impact of UBIs within Vietnam's startup ecosystem. The study employs a mixed-methods approach to systematically evaluate how UBI resources influence startup performance metrics. By examining the interplay between institutional support and entrepreneurial outcomes, the research aims to identify optimal resource allocation strategies that can enhance startup success rates.

	The investigation focuses on four key UBI resources: funding mechanisms, mentorship programs, training initiatives, and networking opportunities. Through quantitative analysis of performance indicators such as revenue growth, innovation output, and market penetration, combined with qualitative insights from successful startup founders, the study provides a nuanced understanding of resource effectiveness. The research methodology incorporates both longitudinal data analysis and case studies to capture the dynamic nature of startup development within UBI environments.

	The findings of this study contribute to both theoretical and practical domains of entrepreneurship research. Theoretically, it advances our understanding of resource-based view in the context of emerging markets and institutional support systems. Practically, the research provides actionable recommendations for UBI program optimization, with evidence suggesting significant improvements in startup performance metrics, as supported by the Vietnam Startup Investment Policy Report (2024). These insights are particularly valuable for policymakers, university administrators, and entrepreneurship educators seeking to enhance the effectiveness of incubation programs in similar emerging market contexts.

	\textbf{Keywords:} incubation program, university-based incubation program, startup
\end{document}