\documentclass[../Main.tex]{subfiles}
\begin{document}
	\section{Thesis Contribution}
	
	This chapter consolidates the research by outlining its primary contributions to theory, practice, and the specific context of Vietnam's entrepreneurial ecosystem. It then provides a summary conclusion of the thesis's core arguments and findings. Finally, it acknowledges the study's limitations and proposes actionable directions for future research to build upon this work.
	
	\subsection{Contributions of the Study}
	
	This research offers significant contributions to the fields of entrepreneurship and strategic management through its theoretical advancements, practical implications for key stakeholders, and contextual insights into developing economies.
	
	\subsubsection{Theoretical Contributions}
	
	The study makes several key contributions to academic theory:
	
	\begin{itemize}
		\item \textbf{Extension of Core Frameworks:} The research extends the application of the Resource-Based View (RBV) and the Technology-Organization-Environment (TOE) framework to the unique context of university-based incubation programs in a developing economy. It demonstrates how these established Western theories can be adapted to explain value creation and performance in emerging entrepreneurial ecosystems.
		
		\item \textbf{Elucidation of a Mediating Mechanism:} By identifying and testing Entrepreneurial Orientation (EO) as a key mediating variable, the study provides a more nuanced understanding of how incubation support is translated into startup performance. It moves beyond a simple input-output model to explain the crucial role of a startup's strategic posture in this process.
		
		\item \textbf{Methodological Advancement:} The thesis proposes and applies a structured, multi-dimensional evaluation model for assessing UBI components and startup performance. This contributes a rigorous and transparent methodological approach that can be used for benchmarking and comparative analysis of incubation programs, addressing a noted lack of standard methodology in the literature.
	\end{itemize}
	
	\subsubsection{Practical and Managerial Contributions}
	
	The findings of this research are designed to provide actionable insights for multiple stakeholders within the innovation ecosystem:
	
	\begin{itemize}
		\item \textbf{For UBI Managers:} The study will guide UBI managers in optimizing program design and resource allocation. By identifying which factors—mentorship, resource access, or networking—most significantly influence startup performance, managers can refine their service offerings to maximize their impact and better support startups.
		
		\item \textbf{For Startup Founders:} This research empowers entrepreneurs to more effectively leverage the resources available within UBIs. It provides a framework for understanding what types of support are most critical for fostering an entrepreneurial orientation and achieving sustainable growth, helping founders make more informed decisions about which programs to join and how to engage with them.
		
		\item \textbf{For Policymakers:} The findings offer evidence-based recommendations for policymakers aiming to foster innovation-driven economies. In a context like Vietnam, with national initiatives such as Project 844 and the National Innovation Center, this research can help strengthen policy design and ensure that public investment in the startup ecosystem yields the greatest possible economic and social returns.
	\end{itemize}
	
	\subsubsection{Contextual Contributions}
	
	The study makes a specific and important contribution by focusing on a previously underexplored area:
	
	\begin{itemize}
		\item \textbf{Addressing a Literature Gap:} This research directly addresses a significant gap in the academic literature concerning the effectiveness and mechanisms of business incubation in developing economies, particularly in the context of Vietnam.
		
		\item \textbf{Providing Nuanced Insights:} It provides detailed insights into the unique challenges and opportunities within Vietnam's startup ecosystem, such as limited access to high-quality mentorship, fragmented networks, and specific resource constraints. This tailored analysis offers far more relevance than the direct application of findings from developed markets.
	\end{itemize}
	
	\section{Conclusion}
	
	This thesis set out to address the core problem of a limited understanding of how University-Based Incubation Programs specifically impact startup performance in Vietnam. The research argues that the effectiveness of these programs hinges on their ability to deliver high-quality, targeted support that fosters a robust Entrepreneurial Orientation within their tenant startups.
	
	The proposed research model posits that key UBI components—namely Mentorship Quality, Resource Access, and Networking Opportunities—do not only have a direct effect on performance but, more importantly, act as catalysts for developing a startup's Entrepreneurial Orientation. This strategic posture, characterized by innovativeness, proactiveness, and risk-taking, is the crucial intermediary that enables startups to transform incubation support into tangible Startup Performance, measured across innovative, market, and financial dimensions.
	
	In essence, the study concludes that the true value of a UBI in a developing context like Vietnam lies less in the provision of physical space and more in its role as a cultivator of entrepreneurial capabilities. By strategically enhancing mentorship, ensuring access to critical resources, and facilitating invaluable network connections, UBIs can empower startups to navigate uncertainty, seize opportunities, and ultimately achieve sustainable growth, thereby making a vital contribution to the nation's innovation agenda.
	
	\section{Limitations and Future Research}
	
	While this study provides significant insights, it is important to acknowledge its limitations, which in turn open up promising avenues for future research.
	
	\subsection{Limitations of the Study}
	
	\begin{itemize}
		\item \textbf{Cross-Sectional Design:} The proposed methodology utilizes a cross-sectional survey design, capturing data at a single point in time. This makes it challenging to establish definitive causality and track the evolution of performance over time.
		
		\item \textbf{Generalizability:} The study's sample is planned to target startups within major entrepreneurial hubs in Vietnam. This focus may limit the generalizability of the findings to startups in more rural or less-developed regions of the country, or to other developing economies with different contextual factors.
		
		\item \textbf{Reliance on Perceptual Data:} The measurement of key constructs like Mentorship Quality and Entrepreneurial Orientation relies partly on self-reported data from founders via Likert scales. This introduces the possibility of common method bias and subjective assessments.
	\end{itemize}
	
	\subsection{Directions for Future Research}
	
	Based on these limitations, the following directions for future research are recommended:
	
	\begin{itemize}
		\item \textbf{Longitudinal Studies:} To better establish causality and understand the long-term impact of UBI support, future research should employ a longitudinal approach. Tracking a cohort of startups from their entry into an incubator through several years post-graduation would provide invaluable data on survival rates and growth trajectories.
		
		\item \textbf{Comparative and Expanded-Scope Studies:} Future studies could expand the research scope to include a more diverse sample, comparing UBIs in major urban centers versus those in rural areas to understand regional disparities. Furthermore, cross-country comparative studies between Vietnam and other developing nations could identify universally effective incubation practices versus those that are context-dependent.
		
		\item \textbf{In-Depth Qualitative Research:} To complement the quantitative findings of this thesis, future work could utilize qualitative methods, such as in-depth case studies. This would allow for a richer exploration of the "why" and "how" behind the observed relationships, including a deeper investigation into the role of informal networks within Vietnam's startup ecosystem.
		
		\item \textbf{Exploration of Other Variables:} Future models could investigate other mediating or moderating variables that may influence the UBI-performance relationship, such as founder characteristics (e.g., resilience, educational background), team dynamics, or the specific impact of government policies like Project 844.
	\end{itemize}
	
\end{document}