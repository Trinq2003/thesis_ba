\documentclass[../Main.tex]{subfiles}
\usepackage{longtable}
\begin{document}
\section{Overview of Factors Affecting Startup Performance in University-Based Incubation Programs}

The foundation of this research rests upon several key concepts that are fundamental to understanding the relationship between university-based incubation programs and startup performance. These concepts form the theoretical framework that guides the investigation into how institutional support mechanisms influence entrepreneurial outcomes within Vietnam's emerging startup ecosystem.

A startup is a newly established business venture that is typically characterized by high growth potential, innovative business models, and scalability. Startups are distinguished from traditional small businesses by their focus on rapid growth, technology-driven solutions, and the potential to disrupt existing markets or create entirely new ones \cite{blank2013startup, ries2011lean}. These ventures often operate in conditions of extreme uncertainty, with unproven business models and untested markets. Startups typically seek to address significant market opportunities through innovative products, services, or processes, and are designed to scale rapidly once product-market fit is achieved. The startup lifecycle typically progresses through several distinct phases: ideation, validation, early growth, and scaling, each requiring different resources, strategies, and support mechanisms \cite{osterwalder2010business}. In the context of university-based incubation, startups often emerge from academic research, student projects, or faculty innovations, representing the commercialization of university-generated knowledge and intellectual property.

Startup performance represents a multifaceted assessment of a new venture's success and effectiveness in achieving its strategic objectives, operational goals, and long-term sustainability within the competitive marketplace. This concept encompasses both quantitative outcomes and qualitative indicators that collectively measure a startup's ability to create value, generate sustainable revenue, attract investment, and contribute to economic development \cite{patton2014realising, barbero2012revisiting}. Unlike traditional business performance metrics, startup performance evaluation must account for the unique characteristics of nascent ventures, including their high uncertainty, rapid evolution, and the critical importance of survival during early stages \cite{mian1996assessing}. The measurement of startup performance requires a comprehensive approach that combines objective quantitative metrics with subjective qualitative assessments, recognizing the complex and evolving nature of entrepreneurial ventures. Survival rate analysis, for instance, is one of the most fundamental and widely accepted measures, with research consistently showing that incubator-supported startups demonstrate significantly higher survival rates compared to non-incubated ventures \cite{NBIA2012}.

University-Based Incubation Programs (UBIs) are initiatives rooted within academic institutions designed to foster the growth of new businesses and startups. These programs leverage the university's resources, including research facilities, faculty expertise, student talent, and networks, to provide a supportive environment for nascent companies. UBIs typically offer a range of services such as workspace, mentorship, access to funding opportunities, business development training, and networking events, aiming to accelerate the success of entrepreneurial ventures emerging from or collaborating with the university \cite{grimaldi2005university, siegel2003assessing}. The effectiveness of these programs is often measured through various performance indicators, including the success rates of incubated startups, the amount of funding raised, and the economic impact generated within the university's ecosystem.

Mentorship within the context of startup development refers to a developmental relationship in which a more experienced or more knowledgeable person helps to guide a less experienced or less knowledgeable person. The mentor acts as a trusted advisor, sharing knowledge, experience, and insights to foster the personal and professional growth of the mentee \cite{jacobi1991mentorship, nationalacademies2019mentoring}. In the startup ecosystem, mentorship is particularly critical as entrepreneurs with mentors are significantly more likely to overcome challenges, pivot effectively, and achieve funding success. Studies have shown that mentored small businesses have significantly higher survival rates compared to non-mentored businesses, highlighting the importance of this support mechanism in entrepreneurial success \cite{kauffman2013mentoring}.

Funding, in the context of startup development, refers to the act of providing financial resources, usually in the form of capital, to support the establishment, operation, or expansion of new ventures. This can come from various sources, including grants, loans, venture capital, angel investors, or crowdfunding \cite{bruneel2010funding, lerner2018venture}. Access to appropriate funding is a fundamental determinant of a startup's ability to launch, grow, and achieve its objectives. The growth of Silicon Valley and the global technology industry exemplifies the transformative power of venture capital funding, where firms like Sequoia Capital and Andreessen Horowitz have provided early-stage capital to companies such as Google, Apple, and Facebook, enabling their massive scale and innovation \cite{lerner2012venture}.

An incubation program is a supportive environment designed to help new and startup businesses develop by providing resources such as workspace, shared services, networking opportunities, business guidance, and sometimes seed funding. These programs typically aim to accelerate the growth and success of fledgling companies by lowering the costs and risks associated with launching a new venture \cite{hackett2004business,europeancommission2014incubators}. Incubators have a proven track record of increasing startup survival rates, with data indicating that incubator graduates have significantly higher success rates compared to non-incubated businesses \cite{NBIA2012}.

Entrepreneurial Orientation (EO) is a concept that captures the strategic orientation of an organization towards entrepreneurship and innovation. It is a framework that helps organizations understand and manage their entrepreneurial behavior, including their willingness to take risks, innovate, and adapt to change. EO is often operationalized through various dimensions, including innovativeness (the organization's ability to generate and implement new ideas and products), proactiveness (the organization's readiness to anticipate and respond to changes in the environment), and risk-taking (the organization's willingness to take calculated risks in pursuit of new opportunities) \cite{wiklund2005entrepreneurial}.

Stakeholders in the context of university-based incubation programs refer to individuals, groups, or organizations that have a vested interest in the success of startup ventures and the effectiveness of incubation programs. These stakeholders can be categorized into primary stakeholders, who are directly involved in the incubation process, and secondary stakeholders, who are indirectly affected by or influence the outcomes. Primary stakeholders include startup founders and teams, university administrators and faculty, mentors and advisors, investors and funding sources, and incubation program managers. Secondary stakeholders encompass government agencies, industry partners, alumni networks, local communities, and the broader entrepreneurial ecosystem \cite{freeman1984strategic, mitchell1997toward}. The effective management of stakeholder relationships is crucial for the success of incubation programs, as different stakeholders provide various resources, expertise, and support mechanisms that contribute to startup development and performance. Understanding stakeholder dynamics and expectations is essential for designing effective incubation programs that can balance the diverse needs and interests of all parties involved.

Innovation capacity refers to an organization's ability to generate, develop, and implement new ideas, products, services, or processes that create value and competitive advantage. In the context of startups, innovation capacity encompasses both the technical ability to develop novel solutions and the organizational capability to bring these innovations to market successfully \cite{lawson2001developing, hurley1998innovation}. This construct includes dimensions such as research and development capabilities, creative problem-solving skills, technological expertise, and the ability to adapt and learn from market feedback. Innovation capacity is particularly critical for startups as it represents their primary source of competitive advantage and differentiation in crowded markets. University-based incubation programs often focus on enhancing startup innovation capacity through access to research facilities, technical expertise, and collaborative opportunities with academic researchers \cite{grimaldi2005university}.

Resource-Based View (RBV) is a strategic management theory that posits that firms can achieve sustainable competitive advantage through the possession of valuable, rare, inimitable, and non-substitutable resources and capabilities \cite{barney1991firm, wernerfelt1984resource}. In the context of university-based incubation, RBV provides a theoretical framework for understanding how access to university resources—including intellectual property, research facilities, faculty expertise, and student talent—can create competitive advantages for incubated startups. The theory suggests that startups that can effectively leverage these unique university resources are more likely to achieve superior performance compared to those that rely solely on generic market resources. RBV also emphasizes the importance of resource heterogeneity and immobility in creating sustainable competitive advantages, which is particularly relevant in the context of university incubators where access to specialized knowledge and facilities can be a significant differentiator \cite{barney2001resource}.

Technology Transfer refers to the process of transferring knowledge, technology, or intellectual property from research institutions (such as universities) to commercial entities (such as startups) for the purpose of creating economic value and societal impact \cite{bozeman2000technology, siegel2003assessing}. In university-based incubation programs, technology transfer plays a crucial role in bridging the gap between academic research and commercial applications. This process typically involves the licensing of patents, copyrights, and trade secrets, as well as the transfer of tacit knowledge through collaborative research projects, consulting arrangements, and personnel exchanges. Technology transfer offices (TTOs) within universities often work closely with incubation programs to identify promising research outcomes and facilitate their commercialization through startup formation. The effectiveness of technology transfer is measured by various metrics including the number of patents filed, licenses granted, startups created, and revenue generated from intellectual property commercialization \cite{phan2005university}.

Entrepreneurial Ecosystem refers to the interconnected network of individuals, organizations, institutions, and resources that support and enable entrepreneurship within a specific geographic region or industry sector \cite{spigel2017relational, stam2015entrepreneurial}. This ecosystem includes various components such as entrepreneurs, investors, mentors, educational institutions, government agencies, support organizations, and infrastructure providers. In the context of university-based incubation, the entrepreneurial ecosystem encompasses the broader network of stakeholders and resources that extend beyond the university campus, including local business communities, industry associations, funding sources, and policy frameworks. The health and vibrancy of the entrepreneurial ecosystem significantly influence the success of incubation programs, as startups require access to diverse resources and networks to grow and scale effectively. Ecosystem development often involves coordinated efforts among multiple stakeholders to create supportive policies, build infrastructure, develop talent, and foster a culture of entrepreneurship \cite{isenberg2011introducing}.

Intellectual Property (IP) refers to creations of the mind, such as inventions, literary and artistic works, designs, symbols, names, and images used in commerce, that are protected by law through patents, copyrights, trademarks, and trade secrets \cite{wipo2020intellectual}. In the context of university-based incubation programs, intellectual property plays a critical role in protecting and commercializing innovative ideas and technologies developed through academic research. Universities often have extensive IP portfolios that can be licensed to startups or used as the foundation for new venture creation. The effective management of intellectual property is crucial for startups as it provides legal protection for their innovations, enhances their competitive position, and can serve as a valuable asset for attracting investment and partnerships. IP strategy in incubation programs typically involves patent filing, trademark registration, copyright protection, and the development of trade secret protocols to safeguard proprietary information and processes \cite{grimaldi2005university, phan2005university}.

The interconnected nature of these concepts forms the theoretical foundation for understanding how university-based incubation programs influence startup performance. The relationship between these elements is complex and multifaceted, with each component playing a crucial role in the overall success of entrepreneurial ventures within academic environments. Understanding these definitions and their practical applications is essential for developing effective strategies to enhance the impact of university-based incubation programs on startup success.

\begin{table}[h]
\centering
\caption{Key Definitions and Examples in University-Based Incubation Research}
\label{tab:key_definitions}
\begin{tabular}{|p{2.5cm}|p{6cm}|p{5cm}|}
\hline
\textbf{Name} & \textbf{Definition} & \textbf{Real-world Examples} \\
\hline
Startup & Newly established business venture characterized by high growth potential, innovative business models, and scalability, distinguished from traditional small businesses by focus on rapid growth and technology-driven solutions \cite{blank2013startup, ries2011lean} & Companies like Uber, Airbnb, and Dropbox that started as innovative solutions and scaled rapidly to become industry disruptors \\
\hline
Startup Performance & Multifaceted assessment of a new venture's success and effectiveness in achieving strategic objectives, operational goals, and long-term sustainability within the competitive marketplace \cite{patton2014realising, barbero2012revisiting} & Y Combinator's success metrics: alumni companies have collectively raised over \$100 billion in funding \cite{YCombinator2024} \\
\hline
University-Based Incubation Programs (UBIs) & Initiatives rooted within academic institutions designed to foster the growth of new businesses and startups, leveraging university resources including research facilities, faculty expertise, student talent, and networks \cite{grimaldi2005university, siegel2003assessing} & Stanford University's StartX: companies have collectively raised billions in capital and created tens of thousands of jobs \cite{StartX2024} \\
\hline
Stakeholders & Individuals, groups, or organizations with vested interest in startup success and incubation program effectiveness, including primary stakeholders (founders, mentors, investors) and secondary stakeholders (government, industry partners, communities) \cite{freeman1984strategic, mitchell1997toward} & University administrators, faculty mentors, angel investors, government agencies, and local business communities all contributing to startup ecosystem \\
\hline
Innovation Capacity & Organization's ability to generate, develop, and implement new ideas, products, services, or processes that create value and competitive advantage \cite{lawson2001developing, hurley1998innovation} & Apple's continuous innovation in product design and user experience, leading to market leadership and high customer loyalty \\
\hline
Resource-Based View (RBV) & Strategic management theory positing that firms achieve sustainable competitive advantage through valuable, rare, inimitable, and non-substitutable resources and capabilities \cite{barney1991firm, wernerfelt1984resource} & Google's proprietary search algorithms and data centers creating barriers to entry and sustaining competitive advantage \\
\hline
Technology Transfer & Process of transferring knowledge, technology, or intellectual property from research institutions to commercial entities for economic value creation \cite{bozeman2000technology, siegel2003assessing} & MIT's Technology Licensing Office facilitating over 800 licenses annually, generating \$200+ million in revenue \cite{mit2024} \\
\hline
Entrepreneurial Ecosystem & Interconnected network of individuals, organizations, institutions, and resources supporting entrepreneurship within specific geographic regions or industry sectors \cite{spigel2017relational, stam2015entrepreneurial} & Silicon Valley ecosystem with Stanford University, venture capital firms, tech companies, and support organizations creating a thriving startup environment \\
\hline
Intellectual Property (IP) & Creations of the mind protected by law through patents, copyrights, trademarks, and trade secrets, including inventions, literary works, designs, and commercial symbols \cite{wipo2020intellectual} & University of California's patent portfolio generating over \$1 billion in licensing revenue, with many patents licensed to startup companies \\
\hline
Mentorship & Developmental relationship in which a more experienced person helps guide a less experienced person, acting as a trusted advisor sharing knowledge, experience, and insights \cite{jacobi1991mentorship, nationalacademies2019mentoring} & Kauffman Foundation study: 70\% of mentored small businesses survive for five years or more, double the rate of non-mentored businesses \cite{kauffman2013mentoring} \\
\hline
Funding & Provision of financial resources, usually in the form of capital, to support establishment, operation, or expansion of new ventures from various sources including grants, loans, venture capital, angel investors, or crowdfunding \cite{bruneel2010funding, lerner2018venture} & Venture capital firms like Sequoia Capital and Andreessen Horowitz providing early-stage capital to companies such as Google, Apple, and Facebook \cite{lerner2012venture} \\
\hline
Incubation Program & Supportive environment designed to help new and startup businesses develop by providing resources such as workspace, shared services, networking opportunities, business guidance, and sometimes seed funding \cite{hackett2004business,europeancommission2014incubators} & National Business Incubation Association data: 87\% of incubator graduates remain in business after three years, compared to approximately 44\% for all businesses \cite{NBIA2012} \\
\hline
Entrepreneurial Orientation (EO) & Strategic orientation of an organization towards entrepreneurship and innovation, including willingness to take risks, innovate, and adapt to change \cite{wiklund2005entrepreneurial} & Dimensions include innovativeness, proactiveness, and risk-taking as key organizational characteristics \\
\hline
\end{tabular}
\end{table}

\section{Literature Review}
\subsection{Technology-Organization-Environment (TOE)}

The Technology-Organization-Environment (TOE) framework represents a comprehensive strategic management approach that emphasizes the dynamic interplay between technological capabilities, organizational structures, and external environmental factors in determining a firm's competitive advantage \cite{toer}. This theoretical framework posits that organizational success is fundamentally determined by the firm's ability to effectively integrate technological innovations into its operational processes, align internal organizational capabilities with technological advancements, and strategically respond to evolving external environmental conditions. The TOE framework provides a holistic perspective that recognizes the interconnected nature of these three critical dimensions in shaping organizational performance and strategic outcomes.

The technological dimension of the TOE framework encompasses the technological capabilities and innovations that a firm possesses or can strategically acquire to enhance its competitive position. This includes both existing technological infrastructure and the organization's capacity to adopt and implement new technologies that can drive innovation and operational efficiency. The organizational dimension focuses on the internal capabilities, structures, and processes that enable a firm to effectively leverage technological resources. This includes factors such as organizational culture, management practices, human capital development, and the alignment of internal processes with technological requirements. The environmental dimension encompasses external factors that influence the organization's ability to leverage technology, including market dynamics, regulatory frameworks, competitive forces, and broader economic conditions that shape the strategic landscape within which organizations operate.

The TOE framework emphasizes the synergistic interaction between these three components in creating sustainable competitive advantages. Technology serves as the foundational element that provides opportunities for innovation and differentiation, while the organization must demonstrate the capability to adapt and align its internal structures and processes to effectively leverage these technological advancements. The external environment acts as both a constraint and an enabler, influencing the organization's strategic choices and determining the effectiveness of technology-organization alignment. This dynamic interplay suggests that successful organizations must continuously balance technological opportunities with organizational capabilities while remaining responsive to environmental changes.

The application of the TOE framework to University-Based Incubation Programs (UBIs) reveals its particular relevance in understanding how these institutions serve as critical nodes within entrepreneurial ecosystems. UBIs provide a unique bundle of technological, organizational, and environmental resources that are often difficult for individual startups to acquire independently. These resources typically include specialized technological infrastructure such as laboratory space and prototyping facilities, access to a vast pool of knowledge including faculty expertise and research findings, and organizational support through mentorship from experienced academics and industry professionals. Additionally, UBIs facilitate environmental connections through networking opportunities with investors and potential partners, thereby addressing the environmental dimension of the TOE framework.

By providing these value-added resources across all three TOE dimensions, UBIs enable startups to overcome initial resource constraints, accelerate their development processes, and significantly improve their chances of survival and innovation \cite{mian1996assessing}. The unique nature and university-specific origin of many of these resources, such as cutting-edge research capabilities and specialized faculty expertise, make them particularly valuable and often inimitable by non-university incubators, thus conferring a distinct competitive advantage to resident startups. This comprehensive resource provision across technological, organizational, and environmental dimensions positions UBIs as strategic facilitators of entrepreneurial success within the broader innovation ecosystem.

\subsection{Resource-Based View (RBV)}

The Resource-Based View (RBV) represents a fundamental strategic management framework that posits that a firm's sustained competitive advantage is derived from the valuable, rare, inimitable, and non-substitutable (VRIN) resources it controls \cite{barney1991firm}. This theoretical perspective represents a significant shift from traditional external industry analysis to a focus on the internal capabilities and assets of an organization as the primary determinants of competitive advantage. Within the RBV framework, resources are broadly defined to encompass tangible assets such as physical infrastructure and financial capital, intangible assets including knowledge, brand reputation, and organizational culture, and organizational capabilities manifested through routines, processes, and skills. A central premise of RBV is that the heterogeneous distribution and imperfect mobility of these resources among firms lead to sustained differences in organizational performance and competitive positioning.

The principles of RBV are particularly relevant and highly applicable to understanding the value proposition of University-Based Incubation Programs (UBIs) in enhancing startup performance. UBIs act as critical resource providers for nascent firms, offering a unique bundle of resources that are often difficult for individual startups to acquire independently \cite{mian1996assessing}. These resources typically include specialized infrastructure such as laboratory space and prototyping facilities, access to a vast pool of knowledge encompassing faculty expertise and research findings, mentorship from experienced academics and industry professionals, and networking opportunities with investors and potential partners. By providing these value-added resources, UBIs enable startups to overcome initial resource constraints, accelerate their development processes, and significantly improve their chances of survival and innovation \cite{mian1996assessing}.

The unique nature and university-specific origin of many of these resources, such as cutting-edge research capabilities and specialized faculty expertise, make them particularly valuable and often inimitable by non-university incubators, thus conferring a distinct competitive advantage to resident startups. This resource uniqueness stems from the academic environment's distinctive characteristics, including access to ongoing research projects, specialized equipment, and intellectual property developed through university research activities. The rarity of these resources is enhanced by the fact that they are typically embedded within the university's institutional context and cannot be easily replicated by commercial incubators or individual entrepreneurs.

An illustrative example of the RBV in action within a UBI context is highlighted by Samsuk and Laosirihongthong \cite{samsuk2014fuzzy}. Their study, using a fuzzy AHP approach, evaluated various enabling factors within incubation programs and identified that resources such as physical infrastructure, including specialized UBI labs, training programs, and mentorship are key drivers of startup performance. Specifically, startups leveraging UBI labs to develop products demonstrated faster growth due to access to equipment and expertise that would otherwise be cost-prohibitive or inaccessible. This empirical evidence directly supports the RBV by demonstrating how the specific resources provided by UBIs, particularly infrastructure and training, serve as critical inputs that directly contribute to the enhanced performance and competitive advantage of the incubated ventures. The study underscores that the strategic deployment and leveraging of these internal resources are fundamental to a startup's success within the UBI ecosystem, validating the core tenets of the Resource-Based View in the context of university-based incubation programs.

\subsubsection{Integration of RBV and Entrepreneurial Ecosystem}

While the Resource-Based View (RBV) and Entrepreneurial Ecosystem Theory offer distinct lenses through which to analyze startup performance, their integration provides a more comprehensive and nuanced understanding of the complex factors influencing entrepreneurial success. The RBV emphasizes the internal strengths derived from a firm's unique resources and capabilities, while the Entrepreneurial Ecosystem Theory highlights the external environment's role in facilitating entrepreneurial activity through interconnected actors and networks. Combining these perspectives acknowledges that a startup's success is not merely a function of its internal assets or external connections in isolation, but rather the synergistic interplay between them. This integrated approach recognizes that both internal resource endowments and external network relationships are essential components of entrepreneurial success, with each dimension reinforcing and amplifying the effectiveness of the other.

An integrated theoretical model posits that startup performance is a direct result of the effective utilization of both internal resources, as highlighted by RBV, and external networks, as emphasized by the Entrepreneurial Ecosystem Theory. Patton \cite{patton2014realising} supports this integration, arguing that University-Based Incubation Programs (UBIs) are uniquely positioned to provide both critical internal resources such as specialized knowledge, infrastructure, and intellectual capital, and extensive external networks connecting startups to investors, mentors, and markets. This dual offering allows UBIs to comprehensively boost startup performance by addressing both the resource-based and network-based dimensions of entrepreneurial success. The value derived from a UBI, therefore, stems from its capacity to bundle valuable resources with crucial relational assets, creating a potent environment for entrepreneurial growth that leverages the strengths of both theoretical perspectives.

The relationship between internal resources and external networks within UBIs is often reciprocal, rather than merely additive, creating a dynamic interplay that enhances overall performance. Barbero et al. \cite{barbero2012revisiting} validate this interaction, demonstrating that a startup's internal resources can enhance its ability to effectively utilize external networks, while strong networks can, in turn, amplify the value of its internal resources. For example, a startup with strong internal technical expertise, which represents a valuable resource according to RBV, may be better equipped to absorb and apply knowledge gained from external mentorship networks facilitated by the entrepreneurial ecosystem. Conversely, access to influential external networks can provide a startup with early market feedback or strategic partnerships, thereby increasing the value and relevance of its existing internal resources, leading to superior performance in terms of growth and innovation.

This dynamic interplay suggests a virtuous cycle where internal strengths enable better external engagement, and external connections strengthen internal capabilities. The integrated model recognizes that the effectiveness of UBI support is maximized when both resource provision and network facilitation are optimized simultaneously, rather than treating them as separate or competing interventions. This perspective provides a more holistic understanding of how UBIs can enhance startup performance by simultaneously addressing the internal resource needs of startups while facilitating their integration into broader entrepreneurial networks. The integrated approach offers valuable insights for UBI managers and policymakers seeking to optimize the design and delivery of incubation services to maximize their impact on startup success.

\subsection{Resource Dependency Theory (RDT)}

Resource Dependency Theory (RDT) provides a fundamental framework for understanding how an organization's strategic behavior is shaped by its dependence on specific resources \cite{pfeffer1978external}. This theoretical perspective suggests that organizations are inherently constrained by the availability and accessibility of critical resources, which in turn fundamentally influences their strategic choices and decision-making processes. RDT emphasizes that organizations' strategic behaviors are directly influenced by their access to critical resources, and that the absence or inadequacy of these resources can significantly limit an organization's ability to achieve its objectives and maintain competitive viability. The theory posits that resource dependencies create power dynamics within organizational relationships, as organizations must navigate their reliance on external resource providers while seeking to maintain strategic autonomy.

The principles of RDT are particularly relevant and highly applicable to University-Based Incubation Programs (UBIs), which serve as critical nodes in the entrepreneurial ecosystem by addressing the resource dependencies of nascent ventures. UBIs provide a unique bundle of resources that are often difficult for individual startups to acquire independently, thereby reducing their resource dependency on external providers. These resources typically include specialized infrastructure such as laboratory space and prototyping facilities, access to a vast pool of knowledge encompassing faculty expertise and research findings, mentorship from experienced academics and industry professionals, and networking opportunities with investors and potential partners. By providing these value-added resources, UBIs enable startups to overcome initial resource constraints, accelerate their development processes, and significantly improve their chances of survival and innovation \cite{mian1996assessing}.

The unique nature and university-specific origin of many of these resources, such as cutting-edge research capabilities and specialized faculty expertise, make them particularly valuable and often inimitable by non-university incubators, thus conferring a distinct competitive advantage to resident startups. This resource provision addresses the fundamental dependency challenges that startups face in their early stages, where access to capital, expertise, and infrastructure can determine their survival and growth potential. The RDT perspective helps explain why startups choose to participate in UBI programs and how these programs can enhance startup performance by reducing resource dependencies and providing access to critical resources that would otherwise be unavailable or prohibitively expensive.

An illustrative example of the RDT in action within a UBI context is highlighted by Samsuk and Laosirihongthong \cite{samsuk2014fuzzy}. Their study, using a fuzzy AHP approach, evaluated various enabling factors within incubation programs and identified that resources such as physical infrastructure, including specialized UBI labs, training programs, and mentorship are key drivers of startup performance. Specifically, startups leveraging UBI labs to develop products demonstrated faster growth due to access to equipment and expertise that would otherwise be cost-prohibitive or inaccessible. This empirical evidence directly supports the RDT by demonstrating how the specific resources provided by UBIs, particularly infrastructure and training, serve as critical inputs that directly contribute to the enhanced performance and competitive advantage of the incubated ventures. The study underscores that the strategic deployment and leveraging of these internal resources are fundamental to a startup's success within the UBI ecosystem, validating the core tenets of Resource Dependency Theory in the context of university-based incubation programs.


\subsection{Entrepreneurial Ecosystem Theory}

The Entrepreneurial Ecosystem Theory represents a comprehensive framework that posits that entrepreneurial activity and startup performance are not solely dependent on individual entrepreneurs or firms, but rather emerge from the complex interactions and relationships among various actors within a supportive regional environment \cite{spigel2017relational}. This theoretical perspective emphasizes the systemic nature of entrepreneurship, recognizing that successful entrepreneurial outcomes are the product of interconnected components working in concert rather than isolated individual efforts. An entrepreneurial ecosystem comprises multiple interconnected components, including formal institutions such as universities and government agencies, informal networks encompassing mentor communities and investor groups, financial capital including venture capital and angel funding, human capital represented by skilled labor and experienced entrepreneurs, and a supportive cultural environment that encourages entrepreneurial activity.

The strength and density of these interconnections facilitate the flow of resources, knowledge, and talent, which are crucial for the creation and growth of new ventures. The theory highlights that a thriving entrepreneurial environment fosters innovation, reduces transaction costs, and provides access to critical resources and opportunities that accelerate startup development. This ecosystem perspective recognizes that entrepreneurial success is not merely a function of individual capabilities but rather the result of the quality and effectiveness of the broader support network within which entrepreneurs operate. The theory suggests that the health and vitality of an entrepreneurial ecosystem can be measured by the density of connections between actors, the quality of resource flows, and the overall capacity of the system to support entrepreneurial activity.

University-Based Incubation Programs (UBIs) serve as pivotal nodes within entrepreneurial ecosystems, acting as key facilitators of interactions and connections among various ecosystem actors. UBIs are instrumental in bridging the gap between startups and vital resources such as investors, industry partners, mentors, and skilled talent \cite{theodorakopoulos2014business}. By actively cultivating and leveraging these networks, UBIs significantly enhance the survival rates and scalability of their resident startups. They provide a structured environment where nascent companies can readily access capital, gain market insights, form strategic alliances, and recruit necessary expertise. This networking function is a cornerstone of a UBI's contribution, ensuring that startups are not isolated but are deeply embedded within a supportive web of relationships that can provide crucial resources and opportunities for growth.

The strength of these UBI-facilitated networks directly contributes to a more robust and dynamic entrepreneurial ecosystem by creating positive feedback loops that benefit all participants. As UBIs successfully connect startups with resources and opportunities, they enhance their own reputation and attractiveness, which in turn draws more high-quality startups and resource providers to the ecosystem. This virtuous cycle strengthens the overall entrepreneurial environment and creates a self-reinforcing system that supports continued innovation and growth. The UBI's role as an ecosystem orchestrator is particularly valuable in emerging entrepreneurial environments where connections between different actors may be weak or nonexistent.

The critical role of UBI networks in boosting startup performance is vividly demonstrated by Harper-Anderson and Lewis \cite{harper2018makes}. Their research highlights how high-quality networks fostered by UBIs enable startups to quickly raise capital and gain essential market access, thereby improving their overall operational performance. For instance, incubator-supported firms often gain expedited access to angel investors and venture capitalists through curated pitch events and introductions facilitated by the UBI's established connections. This direct access to funding significantly reduces the time and effort startups would otherwise expend on fundraising, allowing them to focus more on product development and market penetration. Furthermore, UBI networks provide avenues for market validation, customer acquisition, and partnership formation, which are vital for a startup's early growth. This evidence underscores that the relational capital and brokering capabilities of UBIs, as emphasized by entrepreneurial ecosystem theory, are tangible assets that directly translate into improved financial and operational outcomes for startups.

\subsection{Foundational Evaluation Models and Frameworks for University Business Incubators}

Evaluating the multifaceted performance of University Business Incubators (UBIs) requires a structured and theoretically grounded approach. Over the years, several foundational models have been developed to provide comprehensive frameworks for this complex task, each offering distinct perspectives and measurement criteria that reflect the dual academic and commercial objectives inherent in UBIs.

\subsubsection{Mian's Integrated Framework (1997)}

Sarfraz A. Mian's Integrated Framework, introduced in 1997, stands as a seminal contribution to the assessment and management of University Technology Business Incubators (UTBIs). This framework is distinguished by its comprehensive perspective, integrating three critical performance dimensions that collectively address the sustainability of the incubator, the success of tenant firms, and the broader contributions to the sponsoring university's mission \cite{Chan2005Assessing}. The first dimension, program sustainability and growth, evaluates the incubator's long-term viability and operational efficiency, considering whether the incubator functions as a dynamic and self-sustaining business entity. This includes an examination of management policies and their effectiveness in fostering growth. The second dimension focuses on the survival and growth of tenant firms, assessing metrics such as firm survival rates, employment growth, revenue generation, and the ability to secure venture capital. Mian's framework acknowledges that the needs of technology founders evolve across different stages of development, which complicates the establishment of universal assessment criteria. The third dimension, unique to university-affiliated incubators, considers the incubator's alignment with and contribution to the university's strategic objectives, such as knowledge commercialization, technology transfer, and regional economic development.

In practice, Mian's framework identifies three sets of variables for evaluating effectiveness: performance outcomes, management policies and their effectiveness, and the value-added of services provided. While specific metrics for these variables are not always detailed in the literature, typical value-added services include shared office resources, business assistance, rental incentives, networking opportunities, access to capital, legal and accounting support, and technology-related resources such as laboratory facilities and research and development programs. The flexibility of Mian's framework allows for adaptation to local and regional contexts, a necessity given the diversity of university incubation models worldwide. However, it is important to note that the literature has yet to identify a single, universally effective model for university incubation, underscoring the need for context-specific adaptation and the development of consistent assessment criteria \cite{Yovera2025Academic}.

\subsubsection{Staged Evaluation Frameworks}

Staged evaluation frameworks approach the assessment of university business incubators by dividing the incubation process into distinct phases, each with tailored evaluation criteria. This methodology recognizes that both the needs of startups and the services provided by incubators evolve over time, necessitating different evaluative foci at various points in the incubation journey \cite{Amelia_EvaluationFramework}. The pre-incubation stage emphasizes foundational elements such as infrastructure readiness, the qualifications and experience of the management team, the availability of financial resources, and the clarity of the incubator's vision, mission, and goals. These criteria ensure that the incubator is strategically aligned and adequately prepared to support new ventures.

During the mid-incubation stage, the focus shifts to the effectiveness of program delivery and support services. This includes the mapping and engagement of customer segments, the articulation of a compelling value proposition, the establishment of effective distribution channels, and the development of a robust curriculum. Partnerships with external entities, contributions to economic development, and the facilitation of commercialization and internationalization opportunities are also critical at this stage. The ability to execute business targets and maintain high performance relative to industry standards is closely monitored, as is the integration of technology and standardized processes to enhance productivity and internal cohesion.

The post-incubation stage addresses the support provided to startups after graduation, with an emphasis on enhancing their long-term recognition and success. Evaluation at this stage considers efforts to increase visibility, such as search engine optimization, and the establishment of due diligence processes to ensure legal compliance and commercialization readiness. The enduring impact of the incubation experience is often reflected in the sustained success of alumni startups.

The principal advantage of staged evaluation frameworks lies in their capacity for continuous monitoring and adaptation throughout the incubation lifecycle. By providing granular, phase-specific insights, these frameworks enable incubators to tailor their support services to the evolving needs of their tenants, thereby optimizing effectiveness. However, the comprehensive nature of this approach can be resource-intensive, requiring significant data collection and analytical capacity across multiple time points.

\subsubsection{Institutional Theory Lens}

Institutional theory offers a valuable lens for examining the development and performance of University Incubation Centers (UICs), emphasizing the influence of formal and informal institutions on organizational behavior and outcomes \cite{Kulkarni2024University}. This perspective highlights the importance of legitimacy, regulatory support, and cultural alignment in determining the success of UBIs. Achieving legitimacy—defined as the perception of credibility and competence among stakeholders such as entrepreneurs, investors, academic institutions, and government bodies—is essential for attracting resources and fostering performance outcomes. Organizations that conform to established norms and standards, whether through coercive, mimetic, or normative isomorphism, are more likely to gain acceptance and support within their institutional environment.

Mimetic isomorphism, in particular, describes the tendency of organizations to model themselves after successful counterparts in times of uncertainty. By adopting best practices from established UICs, newer or less established incubators can reduce perceived risk and signal operational effectiveness, thereby enhancing their ability to attract high-quality startups and secure necessary resources. Furthermore, institutional logics—belief systems and practices dominant within academia and industry—shape the strategic orientation of UICs. Successful incubators often balance the academic logic of knowledge creation with the commercial logic of market-oriented innovation, leveraging the strengths of both to foster an environment conducive to entrepreneurial ventures. This balance is especially crucial in dynamic environments where both exploration and exploitation are necessary for sustained success.

The primary benefit of applying an institutional theory lens is its capacity to provide a nuanced, qualitative understanding of UBI effectiveness that extends beyond quantitative metrics. It draws attention to the critical influence of environmental and stakeholder factors, offering insights into how UBIs navigate complex institutional landscapes. However, the qualitative nature of this approach presents challenges in measurement, often necessitating in-depth analysis and interpretation.

\subsubsection{Summary of the Evaluation Models and Frameworks}

\begin{longtable}{|p{0.2\textwidth}|p{0.25\textwidth}|p{0.25\textwidth}|p{0.25\textwidth}|}
    \caption{Summary of the Evaluation Models and Frameworks}
    \label{tab:evaluation-models-frameworks}                                                                                                                                                                                                                                                                                                                                                                                                                                                                                                                                                                                                                                                                                                                        \\
    \hline
    \textbf{Feature / Model}    & \textbf{Mian's Integrated Framework (1997)}                                                                                                             & \textbf{Staged Evaluation Frameworks}                                                                                                                             & \textbf{Institutional Theory Lens}                                                                                                   \\
    \hline
    \endfirsthead
    \caption[]{Summary of the Evaluation Models and Frameworks (Continued)}                                                                                                                                                                                                                                                                                                                                                                                                                                                                                                                                                                                                                                                                                         \\
    \hline
    \textbf{Feature / Model}    & \textbf{Mian's Integrated Framework (1997)}                                                                                                             & \textbf{Staged Evaluation Frameworks}                                                                                                                             & \textbf{Institutional Theory Lens}                                                                                                   \\
    \hline
    \endhead
    \hline
    \multicolumn{4}{r}{\textit{Continued on next page}}                                                                                                                                                                                                                                                                                                                                                                                                                                                                                                                                                                                                                                                                                                             \\
    \endfoot
    \hline
    \endlastfoot
    \textbf{Primary Focus}      & Comprehensive assessment of UTBI performance across program, tenant, and university mission.                                                            & Detailed assessment of incubator effectiveness at distinct phases of incubation.                                                                                  & Understanding UBI performance through legitimacy, norms, and competing logics.                                                       \\
    \hline
    \textbf{Key Dimensions}     & Program sustainability \& growth; Tenant firm survival \& growth; Contributions to university mission.                                                  & Pre-incubation; Mid-incubation; Post-incubation.                                                                                                                  & Legitimacy; Isomorphism (coercive, mimetic, normative); Institutional logics (academic vs. commercial).                              \\
    \hline
    \textbf{Metrics / Criteria} & Performance outcomes, management policies, services \& value-added (specific metrics not detailed in provided sources).                                 & Detailed criteria for each stage (e.g., physical resources, human capital, financial viability, customer engagement, commercialization, post-incubation success). & Qualitative indicators of conformity to norms, adoption of best practices, balance of academic/commercial goals.                     \\
    \hline
    \textbf{Advantages}         & Provides a broad, integrative view of UTBI performance, linking to university mission. Flexible methodology.                                            & Allows for tailored support and evaluation at different developmental stages of startups. Provides granular insights into operational effectiveness.              & Offers a deeper, qualitative understanding of UBI effectiveness, emphasizing external influences and strategic positioning.          \\
    \hline
    \textbf{Limitations}        & Specific metrics for 1997 framework not detailed in provided sources. May not fully capture user perspective. No single effective framework identified. & Can be resource-intensive due to continuous monitoring across multiple stages.                                                                                    & Challenging to quantify and measure, often requiring extensive qualitative analysis.                                                 \\
    \hline
    \textbf{Applicability}      & General assessment of technology-focused university incubators.                                                                                         & Best for detailed operational assessment and tailoring support services throughout the incubation lifecycle.                                                      & Valuable for understanding the strategic positioning and external influences on UBI success, particularly for policy and governance. \\
    \hline
\end{longtable}

\end{document}