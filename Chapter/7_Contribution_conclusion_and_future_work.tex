\documentclass[../Main.tex]{subfiles}
\begin{document}
	\section{Thesis Contribution}
	\label{section:7.1_Thesis_contribution}
	This thesis presents several significant contributions to the field of agentic
	workflow design, AI system integration, and practical application in e-learning
	environments. The contributions can be categorized into primary and secondary
	achievements as follows:
	\subsection{Primary Contributions}
	\label{section:7.1.1_Primary_contributions}
	\begin{itemize}
		\item \textbf{Proposed a Novel Agentic Workflow Creation Framework:} This
			work introduces a new theoretical framework for the design and creation of
			agentic workflows. The framework formalizes the process of constructing flexible,
			modular, and extensible agent workflows, enabling the systematic
			orchestration of reasoning operators and tool integrations. A simple yet functional
			agentic workflow was implemented to demonstrate the practical applicability
			of the proposed framework.

		\item \textbf{Standardized MCP Tool Integration for Agent Systems:} The
			thesis establishes a standardized process and set of technical standards
			for integrating Machine-Callable Protocol (MCP) tools into agent systems. A
			dedicated platform was developed to systematize the onboarding, management,
			and invocation of MCP tools, thereby enhancing interoperability and
			scalability within the agentic ecosystem.

		\item \textbf{Customization of RAGFlow for Vietnamese Document Processing:}
			The open-source RAGFlow framework was extensively customized to support
			Vietnamese language documents. This involved the integration of advanced OCR
			and natural language processing modules tailored for Vietnamese,
			addressing the limitations of the original framework and enabling
			effective retrieval-augmented generation for Vietnamese content.

		\item \textbf{Development of a Centralized Platform for AI Agent Development
			and Testing:} A comprehensive platform was designed and implemented to
			support developers and testers in the creation, debugging, and evaluation
			of AI agent workflows. This platform provides unified interfaces for tool integration,
			workflow management, and real-time monitoring, thereby streamlining the
			development lifecycle of agentic systems.
	\end{itemize}
	\subsection{Secondary Contributions}
	\label{section:7.1.2_Secondary_contributions}
	\begin{itemize}
		\item \textbf{Benchmarking Dataset Generation Workflow:} A workflow was
			developed for the automated generation of benchmarking datasets and test
			cases based on a given knowledge base. This facilitates systematic evaluation
			and validation of agentic workflows and retrieval systems.

		\item \textbf{E-learning Platform Based on AI Agent System:} An e-learning
			platform was created, leveraging the developed AI agent system to provide advanced
			knowledge retrieval, citation, and interactive learning experiences for end
			users.
	\end{itemize}

	\section{Thesis Conclusion}
	\label{section:7.2_Thesis_conclusion}
	In summary, this thesis establishes a structured and comprehensive approach to
	agentic workflow development. The contributions span from the foundational
	proposal of a new agentic workflow design principle, through the architectural
	and practical realization of a full AI agent system and its supporting developer
	tools, to the deployment of an e-learning application built upon these
	innovations. Collectively, these achievements provide a robust methodology and
	technical foundation for the systematic development, integration, and
	application of agentic workflows in complex, real-world environments.

	\section{Future Work}
	\label{section:7.3_Future_work}
	While this thesis has laid a solid foundation for agentic workflow design, AI
	system integration, and practical application in e-learning, several promising
	directions remain for future research and development:
	\begin{itemize}
		\item \textbf{Formal Evaluation and Training of the Agentic Supernet:} The
			current implementation of the agentic workflow utilizes a supernet with
			predefined weights, which does not fully exploit the adaptive and learning
			capabilities envisioned in the proposed framework. Future work will focus
			on training the supernet using appropriate optimization techniques, allowing
			the system to learn optimal operator selection and workflow paths based on
			empirical data. A comprehensive and formal evaluation of the trained
			supernet will be conducted to rigorously assess its performance, generalizability,
			and the practical benefits of the agentic workflow framework.

		\item \textbf{Enhanced Developer Tool Functionality:} To further support
			developers and researchers, the Developer Tool will be extended with advanced
			features for in-depth debugging and analysis of agentic systems. Planned
			enhancements include prompt tracing, which will allow developers to visualize
			and analyze the evolution of prompts throughout the workflow, and memory
			heatmaps, which will provide insights into the agent's memory usage and information
			flow during inference. These features aim to facilitate more effective
			debugging, optimization, and understanding of complex agentic workflows.

		\item \textbf{Advanced RAG Service with Agentic Document Extraction:} The
			RAG service can be further improved by integrating more sophisticated
			document parsing techniques. A promising direction is the adoption of agentic
			document extraction, where specialized agents are employed to extract, structure,
			and annotate information from complex documents. This approach is expected
			to enhance the quality and granularity of the knowledge base, thereby
			improving retrieval accuracy and the overall effectiveness of the RAG
			system.

		\item \textbf{Self-Reinforced Knowledge Library for Dynamic Prompt Injection:}
			Another avenue for future work is the development of a self-reinforced
			knowledge library that can dynamically inject relevant knowledge into each
			operator during agentic workflow inference. This mechanism would enable the
			system to adaptively augment prompts with contextually appropriate information,
			improving reasoning quality, adaptability, and the ability to handle
			diverse and evolving knowledge domains.
	\end{itemize}
	In summary, these future directions aim to advance the theoretical and
	practical capabilities of agentic workflows, enhance developer experience, and
	further improve the performance and adaptability of AI-driven systems in real-world
	applications.
\end{document}