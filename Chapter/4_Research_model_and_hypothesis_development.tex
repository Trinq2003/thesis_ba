\documentclass[../Main.tex]{subfiles}
\usepackage{calc}
\usepackage{longtable}

\begin{document}
	\section{The Proposed  Model for Evaluation: Scoring Framework for UBI Programs}

	\def\arraystretch{1.2} % Increase row height for better readability

\begin{center}
\textbf{SCORING FRAMEWORK FOR THE WORLD BENCHMARK STUDY \cite{Amin2024Incubators,ubi2021world,ubi2019world}}
\end{center}

\begin{figure}[H]
\centering
\caption{The proposed Scoring Framework with its 21 KPIs organized into 7 subcategories and 3 categories}
\label{tab:scoring_framework}
\begin{tabular}{|p{0.25\textwidth}|p{0.3\textwidth}|p{0.35\textwidth}|}
\hline
\textbf{Categories} & \textbf{Subcategories} & \textbf{Dimensions} \\
\hline
\multirow{6}{=}{Value for ecosystem (33.3\%)} & \multirow{4}{=}{Economy enhancement (22.2\%)} & Jobs created \& sustained (6.7\%) \\
\cline{3-3}
& & Sales revenue (6.7\%) \\
\cline{3-3}
& & Graduates (4.4\%) \\
\cline{3-3}
& & Self-generated revenue (4.4\%) \\
\cline{2-3}
& \multirow{2}{=}{Talent retention (11.1\%)} & Client startups accepted (6.7\%) \\
\cline{3-3}
& & Graduate retention (4.4\%) \\
\hline
\multirow{8}{=}{Value for client startups (33.3\%)} & \multirow{2}{=}{Competence development (8.9\%)} & Services offered (4.4\%) \\
\cline{3-3}
& & Coaching and mentoring hours (4.4\%) \\
\cline{2-3}
& \multirow{3}{=}{Access to funds (11.1\%)} & Total investment attracted (6.75\%) \\
\cline{3-3}
& & Average investment attracted (2.2\%) \\
\cline{3-3}
& & Seed funding attraction (2.2\%) \\
\cline{2-3}
& \multirow{3}{=}{Access to network (13.3\%)} & Partners (6.7\%) \\
\cline{3-3}
& & Events (4.4\%) \\
\cline{3-3}
& & Alumni engagement (2.2\%) \\
\hline
\multirow{7}{=}{Value for program (33.3\%)} & \multirow{3}{=}{Program attractiveness (15.5\%)} & In-state applications (6.7\%) \\
\cline{3-3}
& & Out-of-state applications (4.4\%) \\
\cline{3-3}
& & Sponsorship attraction (4.4\%) \\
\cline{2-3}
& \multirow{4}{=}{Post-graduation performance (17.8\%)} & 1-year survival rate (4.4\%) \\
\cline{3-3}
& & 5-year survival rate (4.4\%) \\
\cline{3-3}
& & High growth enterprises (4.4\%) \\
\cline{3-3}
& & Qualified exits (4.4\%) \\
\hline
\end{tabular}
\end{figure}

\subsection{Evaluation Model Description}

The scoring framework presented in Table \ref{tab:scoring_framework} represents a comprehensive evaluation model designed to assess the performance and impact of University Business Incubator (UBI) programs worldwide. This model employs a multi-dimensional approach that captures both quantitative and qualitative aspects of incubator effectiveness through 21 Key Performance Indicators (KPIs) organized into a hierarchical structure.

\subsubsection{Model Structure and Weighting System}

The evaluation model under consideration is structured around three primary categories, each contributing equally with a weight of 33.3\% to the total Program Impact and Performance Score (PIPS). This balanced weighting underscores the comprehensive importance of a UBI's contributions across its various spheres of influence: its broader ecosystem, its direct support to client startups, and its own operational sustainability.

\begin{enumerate}
    \item \textbf{Value for Ecosystem (33.3\%)}: This category assesses the incubator's contribution to the broader economic and social environment. It is composed of two subcategories:
    \begin{itemize}
        \item \textbf{Economy Enhancement (22.2\%)}: Evaluates the direct economic outputs generated by the incubator's activities, including job creation, sales revenue, the number of graduates from the program, and the self-generated revenue of startups.
        \item \textbf{Talent Retention (11.1\%)}: Measures the incubator's success in attracting and retaining human capital, a critical factor for regional innovation, by tracking the number of client startups accepted and the retention of graduates within the ecosystem.
    \end{itemize}
    
    \item \textbf{Value for Client Startups (33.3\%)}: This category focuses on the direct support and resources provided to the incubated ventures. It is divided into three subcategories:
    \begin{itemize}
        \item \textbf{Competence Development (8.9\%)}: Measures the quality and extent of developmental support, such as the variety of services offered and the volume of coaching and mentoring hours provided.
        \item \textbf{Access to Funds (11.1\%)}: Gauges the incubator's effectiveness in facilitating financial support for its startups, assessed through total, average, and seed funding attraction.
        \item \textbf{Access to Network (13.3\%)}: Evaluates the incubator's capacity to provide valuable connections, measured by the number of partners, events hosted, and the level of alumni engagement.
    \end{itemize}
    
    \item \textbf{Value for Program (33.3\%)}: This category evaluates the incubator's own operational success and market position. It consists of two subcategories:
    \begin{itemize}
        \item \textbf{Program Attractiveness (15.5\%)}: Assesses the incubator's reputation and brand strength, reflected in the volume of in-state and out-of-state applications, as well as its ability to attract sponsorship.
        \item \textbf{Post-Graduation Performance (17.8\%)}: Measures the long-term viability and success of its graduated companies, using metrics such as 1-year and 5-year survival rates, the emergence of high-growth enterprises, and the number of qualified exits.
    \end{itemize}
\end{enumerate}

\subsubsection{Key Performance Indicators and Weighting}

The framework is built upon 21 distinct Key Performance Indicators (KPIs), each assigned a specific weight to reflect its relative importance within the overall evaluation. These KPIs are designed to provide a granular view of an incubator's performance across various facets of its operations and impact. The weighting system highlights critical success factors, with indicators such as 'Total investment attracted' (6.75\%), 'Jobs created \& sustained' (6.7\%), 'Sales revenue' (6.7\%), 'Client startups accepted' (6.7\%), 'Partners' (6.7\%), and 'In-state applications' (6.7\%) receiving the highest individual weights. This distribution ensures that the model provides a balanced yet nuanced assessment of an incubator's effectiveness.

\subsection{Reasonableness of the Framework: Alignment with Academic Literature and Global Best Practices}

The proposed UBI Global Scoring Framework demonstrates a high degree of reasonableness, primarily due to its alignment with established academic literature and global best practices in evaluating University Business Incubators. Its structure and the selection of its Key Performance Indicators (KPIs) reflect a mature understanding of the multifaceted roles and impacts of UBIs.

\subsubsection{Holistic and Multi-Dimensional Evaluation}
The tripartite categorization is not an arbitrary division; rather, it directly mirrors the globally recognized, multi-stakeholder objectives of UBIs. This structure ensures a balanced assessment that extends beyond mere financial metrics, capturing the broader societal and developmental roles of UBIs. The equal weighting across these dimensions signifies a deliberate strategic choice to balance economic output, direct client support, and programmatic sustainability. This implies that a UBI's success cannot be singularly defined by its economic contributions if it neglects its responsibilities to its client startups or its own long-term viability. This balanced weighting encourages UBIs to develop comprehensive strategies that address all aspects of their mandate, fostering a more robust and resilient ecosystem. Such an approach allows for a more nuanced understanding of UBI effectiveness, acknowledging that their value extends beyond direct financial returns to include human capital development, network building, and institutional sustainability, all of which are crucial for the long-term health of an innovation ecosystem.

\subsubsection{Validation of Key Performance Indicators (KPIs)}
\paragraph{Value for Ecosystem}
Under the 'Value for Ecosystem' category, KPIs such as "Jobs created \& sustained" (6.7\%) and "Sales revenue" (6.7\%) are consistently highlighted in academic literature as direct and tangible economic outputs of successful incubators \cite{UBI_KeyFactors}. Empirical data from initiatives like the Business Accelerator and Incubator Performance Measurement Framework (BAI PMF) further support this, showing a positive correlation between incubator support and higher employment (14\%) and revenue (13\%) for client companies \cite{FSU_BusIncubator}. Other incubator evaluation frameworks also prioritize "Job Creation" and "Revenue Growth" as top KPIs \cite{Jedhe_Viability}. The inclusion of these highly weighted KPIs under 'Economy Enhancement' is strongly validated by research highlighting UBIs' direct contribution to economic development and employment. This reflects a primary societal expectation from such public-funded initiatives, making these metrics critical accountability measures that demonstrate a direct return on investment for public funds allocated to UBIs.

The 'Talent Retention' subcategory, with KPIs like "Client startups accepted" (6.7\%) and "Graduate retention" (4.4\%), reflects the incubator's crucial role in attracting and retaining human capital. UBI Global itself assesses "alumni engagement" , and academic research identifies "Talent Retention" as a key success factor for UBIs. The emphasis on retaining graduates within the ecosystem is critical for regional innovation and aligns with the broader objective of UBIs fostering human resource development. This subcategory moves beyond immediate economic outputs to capture the long-term human capital development and ecosystem strengthening functions of UBIs. By retaining skilled graduates and successful startups within the local ecosystem, UBIs contribute to a sustainable innovation pipeline, which represents a more subtle but equally vital form of economic enhancement, ultimately contributing to regional competitiveness.

\paragraph{Value for Client Startups}

The 'Value for Client Startups' category is meticulously designed to capture the direct support provided to incubated ventures. "Competence Development" KPIs, such as "Services offered" (4.4\%) and "Coaching and mentoring hours" (4.4\%), reflect the core non-physical services that modern incubators prioritize. Training, consulting, and assistance are consistently identified as crucial non-physical services , and sustained interventions like coaching and mentoring have been empirically linked to improved business performance \cite{Halaby_DeterminingViability,UBI_KeyFactors,Canada_BAIPerformance}.

"Access to Funds" KPIs, including "Total investment attracted" (6.75\%), "Average investment attracted" (2.2\%), and "Seed funding attraction" (2.2\%), are vital for startup survival and growth. Facilitating financial support is a key service expected from UBIs , and "Total capital raised by startups" is a common KPI in other evaluation models \cite{UBI_KeyFactors,QuickersVenture_IncubatorKPI}.

"Access to Network" KPIs, such as "Partners" (6.7\%), "Events" (4.4\%), and "Alumni engagement" (2.2\%), highlight the indispensable importance of network support. A strong network is considered a core component of UBI services , and "Networking capabilities" are recognized as a key performance factor in academic literature. "Strategic Partnerships" is also a widely recognized KPI for incubator success \cite{UBI_KeyFactors,Gerdsri2021Capability}.

The detailed breakdown of 'Competence development,' 'Access to funds,' and 'Access to network' directly reflects the widely recognized "service-centric" nature of modern incubators. In this model, non-physical support—such as mentoring, specialized training, and critical connections—is often more pivotal than merely providing physical space. The significant weighting assigned to 'Access to Network' (13.3\%) and 'Total investment attracted' (6.75\%) validates their direct and profound impact on startup survival and growth. This structure encourages UBIs to actively cultivate and measure the quality of their non-physical support and network, rather than solely reporting on the number of startups housed, thereby shifting the focus from inputs (e.g., facilities) to outcomes (e.g., funding secured, successful partnerships formed).

\paragraph{Value for Program}

The 'Value for Program' category assesses the UBI's operational success and market position. "Program Attractiveness" KPIs, including "In-state applications" (6.7\%), "Out-of-state applications" (4.4\%), and "Sponsorship attraction" (4.4\%), effectively measure the incubator's reputation and its ability to attract both talent and resources, reflecting its overall market standing. The inclusion of "Out-of-state applications" (4.4\%) is particularly noteworthy, as it captures the incubator's ability to attract talent from outside its immediate region, which is a key indicator of its regional and national impact \cite{ubi2019world,ubi2021world}.

"Post-Graduation Performance" KPIs, such as "1-year survival rate" (4.4\%), "5-year survival rate" (4.4\%), "High growth enterprises" (4.4\%), and "Qualified exits" (4.4\%), are crucial for assessing the long-term viability and success of companies that have graduated from the program \cite{UBI_KeyFactors}. Incubators are expected to evolve into self-sustaining entrepreneurship hubs \cite{ubi2019world,ubi2021world}, and tracking the long-term success of their alumni is a key measure of this maturation. "Startup Survival Rate," "Startup Success Rate," and "Exit Value" are also common KPIs in other incubator evaluation frameworks   \cite{QuickersVenture_IncubatorKPI}. This category, particularly 'Post-graduation performance' (17.8\%), addresses the long-term sustainability and ultimate impact of the UBI by tracking the survival and growth of its alumni. This acknowledges that the true measure of an incubator's success extends well beyond the incubation period, necessitating a longitudinal perspective. This focus incentivizes incubators to provide support that builds lasting resilience and growth capacity in their startups, rather than just short-term survival, and encourages investment in robust alumni networks and ongoing support mechanisms.

\paragraph{Strategic Weighting System}
The framework's decision to assign an equal weighting of 33.3\% across the three primary categories is a deliberate strategic choice. This equal distribution underscores the balanced importance of the incubator's contributions to its ecosystem, its client startups, and its own programmatic sustainability. This approach reflects a holistic view of UBI impact, implying that a UBI cannot be considered truly successful if it excels in one area (e.g., generating revenue) but falls short in others (e.g., adequately supporting its startups or ensuring its own long-term sustainability). This balanced weighting encourages UBIs to develop comprehensive strategies that address all aspects of their mandate, fostering a more robust and resilient ecosystem \cite{InsideHigherEd_2023_BestPractices}.

Furthermore, the individual KPI weights (e.g., 'Total investment attracted' at 6.75\%, 'Jobs created \& sustained' at 6.7\%) highlight critical success factors within each subcategory, ensuring a nuanced assessment of performance \cite{ubi2019world,ubi2021world}. This granular weighting aligns with best practices in performance management, which advocate for a focused selection of impactful KPIs that are closely aligned with the organizational mission \cite{InsideHigherEd_2023_BestPractices}.


To further illustrate the alignment, Table \ref{tab:kpis_alignment} provides a comparative overview of the proposed UBI framework's KPIs with concepts and metrics from other academic and industry sources.

\setlength{\LTcapwidth}{\textwidth}
\begin{longtable}{|p{0.3\textwidth}|p{0.34\textwidth}|p{0.34\textwidth}|}
    \caption{Comparative Overview of Proposed UBI Framework's KPIs with Concepts and Metrics from Other Academic and Industry Sources}\label{tab:kpis_alignment} \\
    \hline
    \textbf{Proposed KPI/Subcategory (from User's Framework)} & \textbf{Corresponding Concept/KPI from Academic Literature/Industry} & \textbf{Brief Explanation of Alignment} \\
    \hline
    \endfirsthead
    \hline
    \textbf{Proposed KPI/Subcategory (from User's Framework)} & \textbf{Corresponding Concept/KPI from Academic Literature/Industry} & \textbf{Brief Explanation of Alignment} \\
    \hline
    \endhead
    \hline
    \multicolumn{3}{r}{\textit{Continued on next page}} \\
    \endfoot
    \hline
    \endlastfoot
    Jobs created \& sustained & Economic Boost: Number of jobs created; Employment rate increase & Direct measure of economic impact and job generation, a primary goal of incubators. \\
    \hline
    Sales revenue & Economic Boost: Total revenue for the project; Revenue Growth & Direct measure of economic output and market success of incubated companies. \\
    \hline
    Graduates & Number of graduates; Client success measurement & Tracks the output of the incubation program, indicating successful completion. \\
    \hline
    Self-generated revenue & Incubator revenue sources (activity-based, asset-based, fundraising, investments) & Reflects financial sustainability of the incubator itself or the startups' ability to generate their own income. \\
    \hline
    Client startups accepted & Entry Criteria: Affiliation with the university & Measures the volume and potentially quality of ventures entering the program. \\
    \hline
    Graduate retention & Talent Retention: Effective start for graduates, continuous improvement for graduates & Assesses success in retaining human capital within the ecosystem post-incubation. \\
    \hline
    \multicolumn{3}{|c|}{\textbf{Value for Client Startups}} \\
    \hline
    Services offered & Total service and support; Scope of services; Services quality & Measures the breadth and quality of non-physical support provided to startups. \\
    \hline
    Coaching and mentoring hours & Competency Development: Hours of coaching and mentoring; Mentoring support & Quantifies direct developmental assistance, linked to improved business performance. \\
    \hline
    Total investment attracted & Access to Funds: Attractive total investment; Total capital raised & Critical indicator of financial support facilitation for startups. \\
    \hline
    Average investment attracted & Access to Funds: Interested average investment & Provides context to total investment, indicating typical funding rounds. \\
    \hline
    Seed funding attraction & Access to Funds: Attractiveness of early funding & Focuses on early-stage financial support, crucial for nascent ventures. \\
    \hline
    Partners & Access to Network: Number of partners; Networking capabilities & Measures the strength and reach of the incubator's network for client benefit. \\
    \hline
    Events & Access to Network: Number of events performed & Quantifies opportunities for networking, learning, and exposure. \\
    \hline
    Alumni engagement & Alumni Engaged; Alumni engagement per support & Measures ongoing relationships and value derived from the alumni network. \\
    \hline
    \multicolumn{3}{|c|}{\textbf{Value for Program}} \\
    \hline
    In-state applications & Program reputation and brand strength & Reflects local demand and reputation, indicating program attractiveness. \\
    \hline
    Out-of-state applications & Program reputation and brand strength & Reflects broader national/international appeal and reputation. \\
    \hline
    Sponsorship attraction & Number of interested sponsors; Fundraising & Measures the program's ability to attract external financial support and validation. \\
    \hline
    1-year survival rate & Project survival rate in the first year; Startup Survival Rate & Key long-term performance metric for incubated companies. \\
    \hline
    5-year survival rate & Five-year project survival rate; Startup Survival Rate & Essential for assessing sustained viability and long-term impact. \\
    \hline
    High growth enterprises & Number of companies with high growth rates; Startup Success Rate & Identifies ventures that contribute significantly to economic development. \\
    \hline
    Qualified exits & Exit Value & Measures the successful conclusion of the incubation process through acquisition, IPO, etc. \\
    \hline
\end{longtable}

\subsection{Comparison with Other Evaluation Criteria}
While academic literature notes that "no standard methodology for measuring business incubator performance" exists universally \cite{UBI_KeyFactors}, the proposed framework's KPIs largely align with commonly cited success factors and performance metrics found in various academic studies and industry reports. For example, the Quickers Venture Building Infrastructure, another evaluation model, lists 18 KPIs that show substantial overlap with the proposed framework, including "Startup Survival Rate," "Job Creation," "Revenue Growth," "Mentorship Effectiveness," "Strategic Partnerships," and "Exit Value" \cite{QuickersVenture_IncubatorKPI}.

Academic research also identifies key performance factors for University Incubation Centers (UICs) related to startup performance (e.g., managerial skills, market growth, product innovation, financial health), monitoring and assistance (e.g., marketing, business planning, mentoring, training), and resource munificence (e.g., infrastructure, information/knowledge sharing, networking capabilities) \cite{Kulkarni2024University}. These broadly correspond to the 'Value for Client Startups' category in the proposed framework. Moreover, the seven areas of capability for sustainable development of business incubators—Strategy and Organizational Structure, Finance, Knowledge Body, Human Resource Development, Infrastructure, Network, and Services \cite{Gerdsri2021Capability}.

The consistency of the KPIs across various models, despite the absence of a single "standard" methodology, indicates that the proposed framework is well-grounded. It is not an outlier but rather a well-structured aggregation of recognized best practices, capturing the most commonly accepted and empirically supported dimensions of UBI performance. This consistency strengthens the framework's credibility and its potential for widespread adoption, allowing UBIs using this framework to reasonably compare their performance with global peers, thereby contributing to a more standardized and transparent evaluation landscape.

\section{Variables of the Study}

\subsection{Dependent Variable}
The primary dependent variable in this study is the \textbf{Program Impact and Performance Score (PIPS)}. This is a composite, quantitative measure representing the overall effectiveness, impact, and value generated by a University Business Incubator (UBI). The PIPS is the culminating score derived from the weighted aggregation of 21 distinct Key Performance Indicators (KPIs). It serves as a holistic metric designed to capture the UBI's success across its three core strategic mandates: contributing value to the surrounding ecosystem, providing direct value to its client startups, and ensuring its own programmatic sustainability and success. The operationalization of this variable as a single, comprehensive score allows for a standardized assessment and comparison of different UBI programs.

\subsection{Independent Variables}
The independent variables are the \textbf{21 Key Performance Indicators (KPIs)} that constitute the building blocks of the evaluation framework. These KPIs represent the specific, measurable activities, outputs, and outcomes of the UBI's operations that are hypothesized to influence its overall performance (the PIPS). These variables are direct drivers of performance and can be grouped into the three primary categories of the model:
\begin{enumerate}
    \item \textbf{Value for Ecosystem variables}: These include KPIs that quantify the UBI's external economic and social contributions, such as \textit{Jobs created \& sustained}, \textit{Sales revenue}, and \textit{Talent retention}.
    \item \textbf{Value for Client Startups variables}: This set of variables measures the direct support and resources provided to incubated firms. Examples include \textit{Services offered}, \textit{Coaching and mentoring hours}, \textit{Total investment attracted}, and \textit{Partners}.
    \item \textbf{Value for Program variables}: These variables assess the incubator's own operational health and long-term viability. They include metrics such as \textit{In-state applications}, \textit{Sponsorship attraction}, \textit{5-year survival rate}, and \textit{Qualified exits}.
\end{enumerate}

\subsection{Mediating Variables}
Within the proposed model, the \textbf{seven subcategories} can be conceptualized as mediating variables. These are: \textit{Economy enhancement}, \textit{Talent retention}, \textit{Competence development}, \textit{Access to funds}, \textit{Access to network}, \textit{Program attractiveness}, and \textit{Post-graduation performance}.

These subcategories represent the intermediate mechanisms through which the independent variables (the 21 KPIs) exert their effect on the dependent variable (the PIPS). For instance, the provision of \textit{Coaching and mentoring hours} (an independent variable) does not directly translate into the final PIPS; rather, it first improves the incubator's \textit{Competence development} capability (the mediating variable). This enhanced capability, in turn, contributes to the overall \textit{Value for Client Startups} and subsequently influences the final PIPS. Thus, these subcategories explain the causal pathways by which specific operational activities are transformed into broader, strategic value.

\subsection{Control Variables}
While the framework itself does not specify control variables, a rigorous empirical study comparing the performance of different UBIs using this model would need to account for several contextual factors to ensure a valid comparison. These control variables would help isolate the effect of the incubator's operations (the independent variables) from confounding external influences. Key control variables to consider include:
\begin{itemize}
    \item \textbf{Incubator Characteristics}: Factors such as the \textit{age} and \textit{size} (e.g., number of staff, physical capacity) of the UBI.
    \item \textbf{Geographic and Economic Context}: The \textit{location} of the incubator (e.g., urban vs. rural), the \textit{maturity of the local innovation ecosystem}, and prevailing regional economic conditions.
    \item \textbf{University Affiliation}: The \textit{ranking, size, and research budget} of the parent university, as these can significantly impact resource availability and network access.
    \item \textbf{Industry Focus}: The specific industrial or technological sector the UBI specializes in (e.g., biotechnology, FinTech, software, generalist).
    \item \textbf{Funding Model}: The primary source of the UBI's funding (e.g., public, private, university-funded, or a hybrid model).
\end{itemize}

\section{Hypothesis Development}

\end{document}