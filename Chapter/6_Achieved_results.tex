\documentclass[../Main.tex]{subfiles}
\begin{document}
	\section{RAGFlow Customization for RAG Service}
    \label{section:6.1_ragflow_customization_for_rag_service}

	\begin{figure}[h]
		\centering
		\begin{minipage}[t]{0.45\textwidth}
			\vspace{0pt} % Ensures top alignment
			\small Figure~\ref{fig:ragflow_original_ocr} presents the output generated
			by the original RAGFlow OCR model, which is based on the Deepdoc framework.
			As shown in the image, the recognized text exhibits significant issues when
			processing Vietnamese documents. The extracted content contains numerous
			character errors, missing diacritical marks, and fragmented words, resulting
			in text that is difficult to read and semantically ambiguous. These
			deficiencies are particularly evident in the handling of Vietnamese-specific
			characters and word boundaries, which are essential for accurate information
			retrieval and downstream natural language processing tasks. The observed
			errors clearly demonstrate that the original OCR model is not capable of
			reliably recognizing Vietnamese text, thereby necessitating the integration
			of a more advanced and language-specific OCR solution for effective document
			processing in Vietnamese.
			\vspace{1em}
			In contrast, the Surya-based OCR implementation, as demonstrated in Figure~\ref{fig:ragflow_new_ocr},
			achieved superior performance in Vietnamese document processing. The
			enhanced system successfully preserved diacritical marks, maintained proper
			word boundaries, and accurately recognized Vietnamese-specific characters,
			resulting in significantly improved text quality and readability. This
			substantial improvement in OCR accuracy directly contributes to the
			overall effectiveness of the RAG system by providing higher quality input
			for subsequent processing stages, including embedding generation and information
			retrieval.
		\end{minipage}%
		\hfill
		\begin{minipage}[t]{0.5\textwidth}
			\vspace{0pt} % Ensures top alignment
			\centering
			\includegraphics[width=\textwidth]{Figure/ragflow_original_ocr.png}
			\caption{RAGFlow Original Deepdoc OCR}
			\label{fig:ragflow_original_ocr}
		\end{minipage}
	\end{figure}

	\begin{figure}[h]
		\centering
		\includegraphics[width=\textwidth]{Figure/ragflow_new_ocr.png}
		\caption{RAGFlow New Deepdoc OCR with Surya Integration}
		\label{fig:ragflow_new_ocr}
	\end{figure}

	\begin{figure}[h]
		\centering
		\begin{minipage}[t]{0.45\textwidth}
			\vspace{0pt} % Ensures top alignment
			\centering
			\includegraphics[width=\textwidth]{Figure/ragflow_parsing_result.png}
			\caption{RAGFlow Parsing Result with Surya OCR Benchmark}
			\label{fig:ragflow_parsing_result}
		\end{minipage}%
		\hfill
		\begin{minipage}[t]{0.5\textwidth}
			\vspace{0pt} % Ensures top alignment
			\small The results in Figure~\ref{fig:ragflow_new_ocr} demonstrate that
			the text is well recognized by the Surya-based OCR system, with Vietnamese
			characters, diacritical marks, and word boundaries accurately preserved.
			After applying additional preprocessing steps to the recognized text, the overall
			PDF parsing accuracy was quantitatively evaluated. The evaluation metrics,
			as illustrated in the figure, indicate high levels of precision, recall,
			and F1-score, confirming the effectiveness of the improved OCR and preprocessing
			pipeline for Vietnamese document parsing.
		\end{minipage}
	\end{figure}

	\section{Knowledge Citation Frontend Achieved Results}
    \label{section:6.2_knowledge_citation_frontend_achieved_results}
	\begin{figure}[h]
		\centering
		\includegraphics[width=\textwidth]{Figure/fe_result.jpg}
		\caption{E-learning Frontend System Achieved Results}
		\label{fig:fe_result}
	\end{figure}

	As illustrated in Figure~\ref{fig:fe_result}, a citation box has been successfully
	drawn to localize the position of a referenced chunk within the document. This
	result demonstrates the effectiveness of the Knowledge Citation implementation
	described in Section~\ref{section:5.3.2_programming_document_preview_interface} of
	Chapter~\ref{chapter:Solution_implementation_process}, enabling users to visually
	trace the source of information directly within the document interface.

	\section{MCP Debugging Tool Achieved Result}
    \label{section:6.3_mcp_debugging_tool_achieved_result}
	\begin{figure}[h]
		\centering
		\includegraphics[width=\textwidth]{Figure/mcp_debugging_result.png}
		\caption{MCP Debugging Tool Achieved Result}
		\label{fig:mcp_debugging_result}
	\end{figure}
	Figure~\ref{fig:mcp_debugging_result} demonstrates the successful development of
	an MCP debugging tool within the Developer Tool suite. The image showcases an
	MCP server designed for system information retrieval, which is hosted within an
	n8n Docker container. Notably, this MCP server was generated directly using the
	OpenAPI2MCP Converter, highlighting the effectiveness and practicality of the
	automated conversion process in facilitating seamless integration and
	deployment of new MCP services.

	\section{AI Agent System Achieved Results}
    \label{section:6.4_ai_agent_system_achieved_results}
	\begin{figure}[h]
		\centering
		\includegraphics[width=\textwidth]{Figure/agent_integration_result.png}
		\caption{AI Agent System Achieved Results}
		\label{fig:ai_agent_result}
	\end{figure}
	Figure~\ref{fig:ai_agent_result} demonstrates the successful integration of
	MCP tools into the Tool Ecosystem and the effective connection of the agent
	workflow to the developer tool. The Tool Execution Panel, shown on the right, visualizes
	the tool calling process during agent execution. In this example, the input query
	is “Give me the detailed information of all the running processes that are running
	on the system.” This query is classified as a moderate query and, accordingly,
	is processed through the Chain-of-Thought (CoT) and ReAct operators. The CoT
	operator, functioning as a planner, first drafts a plan to retrieve all
	running processes and then to obtain detailed information for each process. This
	plan is subsequently passed to the ReAct operator, which selects the
	appropriate tools for execution. Specifically, the agent invokes the \texttt{getrunningprocesses}
	tool to retrieve the list of running processes and then utilizes the \texttt{getdetailedprocessinfo}
	tool to gather detailed information for each process. This workflow exemplifies
	the seamless orchestration between planning and tool execution within the agentic
	system.
\end{document}